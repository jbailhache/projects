\documentclass[10pt]{article}
\title{Transfinite ordinals}

\usepackage[left=1cm,right=1cm,top=2cm,bottom=2cm]{geometry}
\usepackage[utf8]{inputenc}
\usepackage{amsfonts}
\usepackage{amssymb}
\usepackage{amsmath}
\usepackage{comment}
\usepackage{verbatim}
\usepackage[T1]{fontenc}
\usepackage{hyperref}

\DeclareUnicodeCharacter{2082}{$_2$}
\DeclareUnicodeCharacter{2083}{$_3$}
\DeclareUnicodeCharacter{2084}{$_4$}
\DeclareUnicodeCharacter{2085}{$_5$}
\DeclareUnicodeCharacter{3A0}{$\Pi$}
\DeclareUnicodeCharacter{3A9}{$\Omega$}
\DeclareUnicodeCharacter{3B3}{$\gamma$}
\DeclareUnicodeCharacter{3B5}{$\epsilon$}
\DeclareUnicodeCharacter{3B8}{$\theta$}
\DeclareUnicodeCharacter{3BE}{$\xi$}
\DeclareUnicodeCharacter{3C0}{$\pi$}
\DeclareUnicodeCharacter{3C8}{$\psi$}


\begin{document}

\title{Commande optimale}
\author{Jacques Bailhache (jacques.bailhache@gmail.com)}

\maketitle

\setlength{\parindent}{0pt}

\section{Multiplicateurs de Lagrange}

On cherche le vecteur \( x = (x_1, x_2, ... x_n) \) de \( \mathbb{R}^n \) tel que l(x) soit maximal tout en satisfaisant les contraintes :
\begin{itemize}
     \setlength{\itemsep}{1pt}
     \setlength{\parskip}{0pt}
     \setlength{\parsep}{0pt}
\item \( h_1(x) = 0 \)
\item \( h_2(x) = 0 \)
\item ...
\item \( h_p(x) = 0 \)
\end{itemize}

On définit le lagrangien L par : 
\[ L(x,p) = l(x) + p_1 h_1(x) + p_2 h_2(x) + ... + p_n h_n(x) \]
avec
\[ p = (p_1, p_2, ... , p_n) \]
On a alors :
\[ \frac{\partial L(x,p)}{\partial x_i} = 0 \]

Pour plus de détails, voir \url{http://log.chez.com/text/math/multiplicateurs_de_lagrange.pdf} .

\section{Commande optimale avec temps discret}

On considère un système dont l'état à un instant donné est représenté par un réel x et dont l'évolution est commandée par un nombre u dont on peut faire varier la valeur à volonté. Plus précisément, dans le cas le plus général, l'état à l'instant t+1 vaut :
\[ x(t+1) = x(t) + f(t,x(t),u(t)) \]
où f est une fonction qui détermine la variation de l'état x en fonction du temps t, de l'état précédent et de la commande u.

Le système passe ainsi par plusieurs états successifs x(1) en t=1, x(2) en t=2, ... jusqu'à x(T) en t=T.

A chaque instant précédant T, on a un certain gain \( l(t,x(t),u(t)) \) et on a aussi un gain final \( m(x(T)) \) qui dépend de l'état final.
On cherche à maximiser la somme des gains à chaque instant plus le gain final, 

Pour simplifier les notations, on pourra écrire f(t) = f(t,x(t),u(t)) et l(t) = l(t,x(t),u(t)).

Considérons par exemple le cas où t = 3.
On a alors :
\begin{itemize}
     \setlength{\itemsep}{1pt}
     \setlength{\parskip}{0pt}
     \setlength{\parsep}{0pt}
\item \( x(1) = x_0 \) donné
\item \( x(2) = x(1) + f(1,x(1),u(1)) \)
\item \( x(3) = x(2) + f(2,x(2),u(2)) \)
\end{itemize}

On cherche alors quelles sont les valeurs de x(1), x(2), x(3), u(1), u(2) qui maximisent l(1,x(1),u(1)) + l(2,x(2),u(2)) + m(x(3)) tout en satisfaisant les contraintes énumérées ci-dessus.




\end{document}