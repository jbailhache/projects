\documentclass[8pt]{article}
\title{Transfinite ordinals}
\usepackage[left=0.5cm,right=0.5cm,top=0.5cm,bottom=0.5cm]{geometry}
\usepackage[utf8]{inputenc}
\usepackage{amsfonts}
\usepackage{amsmath}
\begin{document}

\setlength{\parindent}{0pt}

\vspace{-0.4cm}

\begin{center}
\textbf{TRANSFINITE ORDINALS} by Jacques Bailhache, January-march 2018
\end{center}

\vspace{-0.2cm}
Any ordinal can be defined as the least ordinal strictly greater than all ordinals of a set : the empty set for 0, \(\lbrace \alpha \rbrace\) for the successor of \( \alpha \),  \(\lbrace \alpha_0,\alpha_1,\alpha_2,...\rbrace\) for an ordinal with fundamental sequence \(\alpha_0, \alpha_1, \alpha_2, ...\)

\vspace{-0.7cm}

\section{Algebraic notation}
\vspace{-0.4cm}
We define the following operations on ordinals :
\vspace{-0.4cm}
\smallskip
\begin{itemize}
     \setlength{\itemsep}{1pt}
     \setlength{\parskip}{0pt}
     \setlength{\parsep}{0pt}
\item addition : \( \alpha+0=\alpha ; \alpha+suc(\beta)=suc(\alpha+\beta); \alpha+lim(f)=lim(n \mapsto \alpha+f(n)) \)
\vspace{-0.1cm}
\item multiplication : \( \alpha \times 0 = 0 ; \alpha \times suc(\beta) = (\alpha \times \beta) + \alpha ; \alpha \times lim(f) = lim (n \mapsto \alpha \times f(n)) \)
\vspace{-0.1cm}
\item exponentiation : \( \alpha^0 = 1 ; \alpha^{suc(\beta)} = \alpha^\beta \times \alpha ; \alpha^{lim(f)} = lim (n \mapsto \alpha^{f(n)}) \)
\end{itemize}

\vspace{-0.8cm}

\section{Veblen functions}
\vspace{-0.4cm}

\( \varepsilon_0 = lim\ \omega, \omega^\omega, \omega^{\omega^\omega}, \ldots ; \varepsilon_1 = lim\ \varepsilon_0, {\varepsilon_0}^{\varepsilon_0}, {\varepsilon_0}^{{\varepsilon_0}^{\varepsilon_0}}, \ldots = lim\ \varepsilon_0+1, \omega^{\varepsilon_0+1}, \omega^{\omega^{\varepsilon_0+1}}, \ldots ; \zeta_0 = lim\ 0, \varepsilon_0, \varepsilon_{\varepsilon_0}, \ldots \)

\( \omega^\alpha = \varphi_0(\alpha) = \varphi(0,\alpha) ; \varepsilon_\alpha = \varphi_1(\alpha) = \varphi(1,\alpha) ; \zeta_\alpha = \varphi_2(\alpha) = \varphi(2,\alpha) \)

\(\varphi(\ldots,\beta,0,\ldots,0,\gamma) \) is the \((1+\gamma)^{th}\) common fixed point of the functions \( \xi \mapsto \varphi(\ldots,\delta,\xi,0,\ldots,0)\) for all \(\delta < \beta\).

\vspace{-0.2cm}

\( \varphi(\alpha_n,\ldots,\alpha_0,\beta) \) may also be written \( \varphi_{\alpha_n,\ldots,\alpha_0}(\beta) \) or \( \varphi_{\Omega^n\times\alpha_n+\ldots+\alpha_0}(\beta) \) or \( \varphi(\Omega^n\times\alpha_n+\ldots+\alpha_0,\beta) \) or \(
\begin{pmatrix}
    \beta & \alpha_0 & \ldots & \alpha_n \\
    0     & 1        & \ldots & n+1
\end{pmatrix} \)

\vspace{-0.8cm}

\section{Simmons notation}
\vspace{-0.4cm}
\( Fix f z = f^w(z+1)\) = least fixed point of f strictly greater than z ; \( Next = Fix (\alpha \mapsto \omega^\alpha) \) 

\( [0] h = Fix (\alpha \mapsto h^\alpha \omega) \) ;
\( [1] h g = Fix (\alpha \mapsto h^\alpha g \omega) \) ;
\( [2] h g f = Fix (\alpha \mapsto h^\alpha g f \omega) \) ; etc...

Correspondence with Veblen's \(\varphi\) : \( \varphi(1+\beta,\alpha) = ([0]^\beta Next)^{1+\alpha} \omega \)

If \( \gamma > 0 , \varphi(\gamma,\beta,\alpha) = \varphi(\gamma\times\Omega+\beta,\alpha) = ([0]^{\gamma\times\Omega+\beta} Next)^{1+\alpha} \omega = ([0]^\beta (([0]^\Omega)^\gamma Next))^{1+\alpha} \omega = ([0]^\beta (([1] [0])^\gamma Next))^{1+\alpha} \omega \)

If \( \delta > 0 \) or \( \gamma > 0, \varphi(\delta,\gamma,\beta,\alpha) = \varphi(\delta\times\Omega^2+\gamma\times\Omega+\beta,\alpha) = ([0]^{\Omega^2\times\delta+\Omega\times\gamma+\beta} Next)^{1+\alpha} \omega = ([0]^\beta ([0]^\Omega)^\gamma ([0]^{\Omega^2})^\delta Next)))^{1+\alpha} \omega = ([0]^\beta (([1] [0])^\gamma (([1]^2 [0])^\delta Next)))^{1+\alpha} \omega \), with \( [0]^{\Omega^n} = [1]^n [0] \).

Rationalization of \( \varphi : \varphi(1+\beta,\alpha) = \varphi'(\beta,1+\alpha) => \varphi'(\beta,\alpha) = ([0]^\beta Next)^\alpha \omega ; \varphi(\gamma,\beta,\alpha) = \varphi'(\gamma,\beta,1+\alpha) \) 

\vspace{-0.6cm}

\section{RHS0 notation}
\vspace{-0.4cm}
We start from 0, if we don(t see any regularity we take the successor, if we see a regularity, if we have a notation for this regularity, we use it, else we invent it, then we jump to the limit.

\( H f x = lim\ x, f x, f (f x), \ldots ; R_1 f g x = lim\ g x, f g x, f f g x, \ldots ; R_2 f g h x = lim\ h x, f g h x, f g f g h x, \ldots \)

Correspondence with Simmons notation : 
\( \ldots, [3] \rightarrow R_5, [2] \rightarrow R_4, [1] \rightarrow R_3, [0] \rightarrow R_2, Next \rightarrow R_1, \omega \rightarrow H suc\ 0 \)

\vspace{-0.6cm}

\section{Tree ordinals}

\vspace{-0.4cm}

A tree ordinal a belongs to the tree ordinal class \( \Omega_n (n \in \mathbb{N} \) if either a = 0, a = a' + 1 for some tree ordinal a' belonging to the tree ordinal class \( \Omega_n \), or a is a function from \( \Omega_k \) to \( \Omega_n \) for some k < n.

To any tree ordinal a, we can associate a corresponding ordinal \( \alpha = |a| \) obtained by ignoring the choice of particular fundamental sequences, and defined by :
 \( |0| = 0 \) ;
 \( |a+1| = |a|+1 \) ;
 \( |a| = sup |a[b]| \) if a is a function from \( \Omega_k \) to \( \Omega_n \).

We can define the following extension of the Fast Growing Hierarchy (which corresponds to the case n=0) :

\vspace{-0.4cm}

\begin{itemize}
     \setlength{\itemsep}{1pt}
     \setlength{\parskip}{0pt}
     \setlength{\parsep}{0pt}
\item \( F_n(0,b) = b+1 \)
\item \( F_n(a+1,b) = [F_n(a,\bullet)]^b(b) \)
\item \( (F_n(a,b))[c] = F_n(a[c],b) \) if a is a function from \( \Omega_k \) to \( \Omega_{n+1} \) with \( k < n \)
\item \( (F_n(a,b)) = F_n(a[b],b) \) if a is a function from \( \Omega_n \) to \( \Omega_{n+1} \)
\end{itemize}

\vspace{-0.6cm}

\section{Ordinal collapsing functions}
\vspace{-0.4cm}
These functions use uncountable ordinals to define countable ordinals.

We define sets of ordinals that can be built from given ordinals and operations, then we take the least ordinal which is not in this set, or the least ordinal which is greater than all contable ordinals of this set.

These functions are extensions of functions on countable ordinals, whose fixed points can be reached by applying them to an uncountable ordinal, for example :

\vspace{-0.4cm}
\smallskip
\begin{itemize}
     \setlength{\itemsep}{1pt}
     \setlength{\parskip}{0pt}
     \setlength{\parsep}{0pt}

\item Buchholz \( \psi_0 \) : \( \psi_0(\alpha) = \omega^\alpha \) if \( \alpha < \varepsilon_0 ; \psi_0(\Omega) = \varepsilon_0 \) which is the least fixed point of \( \alpha \mapsto \omega^\alpha \).
\vspace{-0.1cm}

\item Madore's \(\psi\) : \(\psi(\alpha) = \varepsilon_\alpha \) if \(\alpha < \zeta_0 ; \psi(\Omega) = \zeta_0 \) which is the least fixed point of \( \alpha \mapsto \varepsilon_\alpha \).
\vspace{-0.1cm}

\item Feferman's \(\theta\) : \(\theta(\alpha,\beta) = \varphi(\alpha,\beta) \) if \( \alpha < \Gamma_0 \) and \( \beta < \Gamma_0 ; \theta(\Omega,0) = \Gamma_0 \) which is the least fixed point of \( \alpha \mapsto \varphi(\alpha,0) \).
\vspace{-0.1cm}

\item Taranovsky's C : \( C(\alpha,\beta) = \beta+\omega^\alpha \) if \( \alpha \) is countable; \( C(\Omega_1,0) = \varepsilon_0 \) which is the least fixed point of \( \alpha \mapsto \omega^\alpha \).

\end{itemize}

Some general formulas for ordinal collapsing functions are : 

\begin{itemize}
     \setlength{\itemsep}{1pt}
     \setlength{\parskip}{0pt}
     \setlength{\parsep}{0pt}

\item \( \psi_\nu(0) = z(\nu) \) ( for example : \( \psi_\nu(0) = \Omega_\nu \), or \( \psi_0(0) = 1; \psi_{1+\nu}(0) = \Omega_{1+\nu} = \omega_{1+\nu} \) 

\item \( \psi_\nu(suc\ \alpha) = f(\psi_\nu(\alpha)) \) 

\item \( \psi_\nu(lim\ h) = lim(\psi_\nu \circ h) \) ( with \( lim = Lim_0 \) )

\item \( \psi_\nu(Lim_{\kappa+1} h) = Lim_{\kappa+1}(\psi_\nu \circ h) \) if \( \kappa < \nu \), 
 or with fundamental sequence notation : \( \psi_\nu(\alpha)[\eta] = \psi_\nu(\alpha[\eta]) \)

\item \( \psi_\nu (Lim_{\kappa+1} h) = lim [ \psi_\nu (h ((\psi_\kappa \circ h)^\bullet (\zeta)))] \) if \( \kappa \ge \nu \), with \( \zeta = 0 \) or 1 or \( \psi_\kappa(0) \) for example.

\end{itemize}


\vspace{0.1cm}

\begin{tabular}{|c|c|c|c|c|c|c|c|c|}
\hline
Name		& Symbol		& Algebraic			& Veblen			& Simmons			& RHS0 		& Madore				& Taranovsky 			\\
\hline
Zero		& 0			& 0				& 				& 				& 0			& 					& 0				\\ \hline
One		& 1			& 1				& \(\varphi(0,0)\)		& 				& suc 0			& 					& C(0,0)			\\ \hline
Two		& 2			& 2				& 				& 				& suc (suc 0)		& 					& C(0,C(0,0))			\\ \hline
Omega		& \(\omega\)		& \(\omega\)			& \(\varphi(0,1)\)		& \(\omega\)			& H suc 0		& 					& C(1,0)			\\ \hline
		& 			& \(\omega+1\)			& 				& 				& suc (H suc 0)		& 					& C(0,C(1,0))			\\ \hline
		&			& \(\omega\times2\)		&				& 				& H suc (H suc 0)	& 					& C(1,C(1,0))			\\ \hline
		&			& \(\omega^2\)			& \(\varphi(0,2)\)		& 				& H (H suc) 0		& 					& C(C(0,C(0,0)),0)		\\ \hline
		&			& \(\omega^\omega\)		& \(\varphi(0,\omega)\)		& 				& H H suc 0		& 					& C(C(1,0),0)			\\ \hline
Epsilon zero	& \(\varepsilon_0\)	& \(\varepsilon_0\)		& \(\varphi(1,0)\)		& \(Next\ \omega\)		& \(R_1 H suc\ 0\)	& \(\psi(0)\)				& \(C(\Omega_1,0)\)		\\ \hline
		& 			& \(\varepsilon_1\)		& \(\varphi(1,1)\)		& \(Next^2 \omega\)	& \(R_1 (R_1 H) suc\ 0\)& \(\psi(1)\)				& \(C(\Omega_1,C(\Omega_1,0)\)	\\ \hline
		& 			& \(\varepsilon_\omega\)	& \(\varphi(1,\omega)\) 	& \(Next^\omega \omega\) & \(H R_1 H suc\ 0\)	& \(\psi(\omega)\)			& \(C(C(0,\Omega_1),0)\)	\\ \hline
		& 			&\(\varepsilon_{\varepsilon_0}\)& \(\varphi(1,\varphi(1,0))\)	& \(Next^{Next \omega} \omega \) & \(R_1 H R_1 H suc\ 0\)& \(\psi(\psi(0))\)			& \(C(C(C(\Omega_1,0),\Omega_1),0)\)\\ \hline
Zeta zero	& \(\zeta_0\)		& \(\zeta_0\)			& \(\varphi(2,0)\)		& \([0] Next\ \omega\)		& \(R_2 R_1 H suc\ 0\)	& \(\psi(\Omega)\)			& \(C(C(\Omega_1,\Omega_1),0)\)	\\ \hline
Eta zero	& \(\eta_0\)		& \(\eta_0\)			& \(\varphi(3,0)\)		& \([0]^2 Next\ \omega\) 	& \(R_2 (R_2 R_1) H suc\ 0\)&					& \(C(C(\Omega,C(\Omega,\Omega)),0)\) \\ \hline
		&			&			& \(\varphi(\omega,0)\)		& \([0]^\omega Next\ \omega\) & \(H R_2 R_1 H suc\ 0\)&					& \(C(C(C(0,\Omega_1),\Omega_1),0)\) \\ \hline
Feferman	& \(\Gamma_0\)		
								& \(\Gamma_0\)			& \(\varphi(1,0,0)\)		& \([1] [0] Next\ \omega\)	& \(R_3 R_2 R_1 H suc\ 0\) & \(\psi(\Omega^\Omega)\)		& \(C(C(C(\Omega_1,\Omega_1),\) \\ 
-Schütte	&			&				& \(=\varphi(2 \mapsto 1)\)	&				& \(= R_{3 \ldots 1} H suc\ 0\) & 					& \(\Omega_1),0)\)		\\ \hline
Ackermann	&			&				& \(\varphi(1,0,0,0)\)		& \([1]^2 [0] Next\ \omega\) & \(R_3 (R_3 R_2) R_1 H suc\ 0\) & \(\psi(\Omega^{\Omega^2})\)		&				\\ 
		&			&				& \(=\varphi(3 \mapsto 1)\)	&				&			&					&				\\ \hline
Small Veblen	&			&				& \(\varphi(\omega \mapsto 1)\)	& \([1]^\omega [0] Next\ \omega\) & \(H R_3 R_2 R_1 H suc\ 0\) & \(\psi(\Omega^{\Omega^\omega})\)	& \(C(\Omega_1^\omega,0)\)	\\
ordinal		&			&				&				&				&			&					& \(=C(C(C(C(0,\Omega_1), \)	\\ 
		&			&				&				&				&			&					& \(\Omega_1),\Omega_1),0)\)	\\ \hline
Large Veblen	&			&				& least ord.	 	 	& \([2] [1] [0] Next\ \omega\)	& \(R_4 R_3 R_2 R_1 H suc\ 0\) & \(\psi(\Omega^{\Omega^\Omega})\)	& \(C(\Omega_1^{\Omega_1},0)\)	\\
ordinal		&			&				& not rep.			&				& \(= R_{4 \ldots 1} H suc\ 0\) &					& \(=C(C(C(C(\Omega_1,\Omega_1),\) \\ 
		&			&				&				&				&			&					& \( \Omega_1),\Omega_1),0) \)	\\ \hline
Bachmann-	&			&				&				& least ord.			& \(R_{\omega \ldots 1} H suc\ 0\) & \(\psi(\varepsilon_{\Omega+1})\)	& \(C(C(\Omega_2,\Omega_1),0)\)	\\
Howard		&			&				&				& not rep.			&			&					&				\\ \hline
  
\end{tabular}


\end{document}

