\documentclass[10pt]{article}
\title{Transfinite ordinals}

\usepackage[left=1cm,right=1cm,top=2cm,bottom=2cm]{geometry}
\usepackage[utf8]{inputenc}
\usepackage{amsfonts}
\usepackage{amssymb}
\usepackage{amsmath}
\usepackage{comment}

\begin{document}

\title{A Tutorial Overview of Ordinal Notations}
\author{Jacques Bailhache (jacques.bailhache@gmail.com)}

\maketitle

\setlength{\parindent}{0pt}

ocfrecdef.txt

10.1

\( \psi_\nu (Lim_{\kappa+1} h) = lim [ \psi_\nu (h ((\psi_\kappa \circ h)^\bullet (\zeta)))] \)

Concerning the last formula, with ... we also get the previous one for \( \psi_0 \) but it is not the same formula as for Buchholz function which we will see later.


buchholz psi functions
10.3.3 7.

\( \psi_\nu(Lim_{\mu+1} h) = lim (\xi \mapsto \psi_\nu (h ((\psi_\mu \circ h)^\xi (\Omega_\mu)))) \) 

\bigskip

Maksudov

first system

\bigskip

3) \( \psi_{\chi(0,0)}(0) = 1 \)

4) \( \psi_{\chi(0,\beta+1)}(0) = \chi(0,\beta) \cdot \omega \)

5) \( \psi_{\chi(0,\beta)}(\gamma+1) = \psi_{\chi(0,\beta)}(\gamma) \cdot \omega \)

\bigskip

6) \( \psi_{\chi(\beta+1,0)}(0) = [\chi(\beta,\bullet)]^\omega (0) \)

7) \( \psi_{\chi(\beta+1,\gamma+1)}(0) = [\chi(\beta,\bullet)]^\omega (\chi(\beta+1,\gamma)+1) \)

8) \( \psi_{\chi(\beta+1,\gamma)}(\delta+1) = [\chi(\beta,\bullet)]^\omega (\psi_{\chi(\beta+1,\gamma)}(\delta)+1) \)

\bigskip

9) \( \psi_{\chi(Lim_\mu f,0)}(0) = Lim_\mu [\chi(f(\bullet),0)] \) if \( \omega_\mu \ge \omega \)

10) \( \psi_{\chi(Lim_\mu f,\gamma+1)}(0) = Lim_\mu [\chi(f(\bullet),chi(Lim_\mu f,\gamma)+1))] \) if \( M > \omega_\mu \ge \omega \)

11) \( \psi_{\chi(Lim_\mu f,\gamma)}(\delta+1) = Lim_\mu [\chi(f(\bullet),\psi_{\chi(Lim_\mu f,\gamma)}(\delta)+1)] \)

\bigskip

12) \( \psi_{\chi(lim_M f,0)}(0) = [\chi(f(\bullet),0]^\omega (1) \)

13) \( \psi_{\chi(lim_M f,\gamma+1)}(0) = [\chi(f(\bullet),0]^\omega (\chi(Lim_M f,\gamma)+1) \)

14) \( \psi_{\chi(lim_M f,\gamma)}(\delta+1) = [\chi(f(\bullet),0]^\omega (\psi_{\chi(Lim_M f,\gamma)}(\delta)+1) \)

\bigskip

18) \( \chi(\beta,Lim_\mu f) = Lim_\mu [\chi(\beta,f(\bullet))] \) if \( \omega_\mu \ge \omega \)

19) \( \psi_\pi(Lim_\mu f) = Lim_\mu (\psi_\pi \circ f) \) if \( \pi > \omega_\mu \ge \omega \)

20) \( \psi_\pi(Lim_\mu f) = lim [ \psi_\pi(f((\psi_\mu \circ f)^\bullet(1))) ] \) if \( \omega_\mu \ge \pi \)

\bigskip

second system

\bigskip

7) \( \psi_{\chi(0)}(0) = 1 \)

2) \( \psi_{\chi(\beta+1)}(0) = \chi(\beta) \cdot \omega \)

3) \( \psi_{\chi(Lim_\mu f)}(0) = Lim_\mu(\chi \circ f) \) if \( \omega_\kappa < M \)

4) \( \psi_{\chi(lim_M f)}(0) = (\chi \circ f)^\omega (1) \)

6) \( \psi_{\chi(\lim_M f)}(\gamma+1) = (\chi \circ f)^\omega (\psi_{\chi(\beta)}(\gamma)+1) \)

5) \( \psi_{\chi(\beta)}(\gamma+1) = \psi_{\chi(\beta)}(\gamma) \cdot \omega \)

11) \( \psi_\pi (Lim_\mu f) = Lim_\mu (\psi_\pi \circ f) \) if \( \pi > \omega_\mu \ge \omega \)

12) \( \psi_\pi (Lim_\mu f) = lim [\psi_\pi(f((\psi_\mu \circ f)^\bullet(1)))] \) if \( \omega_\mu \ge \pi \)


\end{document}

