\documentclass[10pt]{article}
\title{Transfinite ordinals}

\usepackage[left=1cm,right=1cm,top=2cm,bottom=2cm]{geometry}
\usepackage[utf8]{inputenc}
\usepackage{amsfonts}
\usepackage{amssymb}
\usepackage{amsmath}
\usepackage{comment}
\usepackage{verbatim}
\usepackage[T1]{fontenc}

\DeclareUnicodeCharacter{2082}{$_2$}
\DeclareUnicodeCharacter{2083}{$_3$}
\DeclareUnicodeCharacter{2084}{$_4$}
\DeclareUnicodeCharacter{2085}{$_5$}
\DeclareUnicodeCharacter{3A0}{$\Pi$}
\DeclareUnicodeCharacter{3A9}{$\Omega$}
\DeclareUnicodeCharacter{3B3}{$\gamma$}
\DeclareUnicodeCharacter{3B5}{$\epsilon$}
\DeclareUnicodeCharacter{3B8}{$\theta$}
\DeclareUnicodeCharacter{3BE}{$\xi$}
\DeclareUnicodeCharacter{3C0}{$\pi$}
\DeclareUnicodeCharacter{3C8}{$\psi$}


\begin{document}

\title{Ordinal Collapsing Functions and Taranovsky Ordinal Notation}
\author{Jacques Bailhache (jacques.bailhache@gmail.com)}

\maketitle

\setlength{\parindent}{0pt}

\section{Ordinal collapsing functions and Taranovsky Ordinal Notation}

One way to define large countable ordinals is to collapse a large (uncountable) cardinal with an ordinal collapsing function (OCF).

In http://www.madore.org/~david/math/ordinal-zoo.pdf David Madore enumerates some examples of ordinals defined this way, as the countable collapse of $epsilon_{\kappa+1}$ (which is the limit of $\kappa, \kappa^\kappa, \kappa^{\kappa^\kappa}, ...$) where $\kappa$ is respectively :

\begin{itemize}
 \item the first inaccessible cardinal I
 \item the first Mahlo cardinal M
 \item the first weakly compact cardinal K
 \item the first $\Pi^0_2$ indescribable cardinal
\end{itemize}

 The definition of large cardinals is very complex, as shown by David Madore's blog post 
 
 http://www.madore.org/~david/weblog/d.2017-08-31.2462.ordinaux-interessants.html :
 
\bigskip

\newenvironment{changemargin}[2]{\begin{list}{}{%
\setlength{\topsep}{0pt}%
\setlength{\leftmargin}{0pt}%
\setlength{\rightmargin}{0pt}%
\setlength{\listparindent}{\parindent}%
\setlength{\itemindent}{\parindent}%
\setlength{\parsep}{0pt plus 1pt}%
\addtolength{\leftmargin}{#1}%
\addtolength{\rightmargin}{#2}%
}\item }{\end{list}}

\begin{changemargin}{1cm}{1cm}
 
	Pour aller encore plus loin, on fait appel à un cardinal « faiblement compact » K, une notion de grand cardinal (domaine (3)) dont l'analogue récursif (domaine (2)) s'appelle « Π₃-réfléchissant », quelque chose que je ne compte pas définir, mais à ce niveau-là, le système de fonctions d'écrasement devient encore plus complexe. Au niveau de l'écrasement d'un cardinal récursivement Mahlo (:= Π₂-réfléchissant sur les admissibles, c'est-à-dire Π₂-réfléchissant sur les Π₂-réfléchissants), que j'avais évoqué ci-dessus, il y avait deux « niveaux » de fonctions d'écrasement : θ qui sert à fabriquer des cardinaux réguliers / ordinaux admissibles π<M, et toutes sortes de fonctions ψπ qui fabriquent des ordinaux <π. Maintenant qu'on veut jouer avec K, les choses se compliquent encore : on a une hiérarchie par niveau de « Mahloïtude », avec au niveau 0 tous les ordinaux <K, au niveau 1 les cardinaux réguliers ou ordinaux admissibles, au niveau 2 les cardinaux Mahlo ou ordinaux récursivement Mahlo, et au niveau γ+1 les cardinaux réguliers dans lesquels ceux du niveau γ sont clos-cofinaux (ou quelque chose d'analogue côté récursif : les ordinaux Π₂-réfléchissant sur ceux de niveau γ) ; on a une fonction d'écrasement qui produit le plus petit élément de chaque niveau de Mahloïtude, mais surtout, on a en a une autre ψξπ de trois variables qui prend un niveau de Mahloïtude ξ, un π dans lequel écraser (qui est lui-même d'un certain niveau de Mahloïtude >ξ), et un ordinal à écraser, et qui produit un écrasement <π de niveau de Mahloïtude ξ (qui peut donc servir pour de nouveaux écrasements si ξ>0). Bref, là où au paragraphe précédent on avait juste le niveau 0 (la fonction ψ) et une ébauche de niveau 1 (la fonction θ), on a maintenant toute une hiérarchie de niveaux, et K peut servir à diagonaliser sur la hiérarchie elle-même (produire le plus petit cardinal dont le niveau de Mahloïtude est égal à lui-même, etc.).

	L'ordinal ψ0Ω(εK+1) décrit par tout ce fatras mesure la force d'une théorie qui est la théorie des ensembles de Kripke-Platek enrichie d'un axiome de Π₃-réflexion.

	Tout ça est déjà bien compliqué. Chronologiquement, cette analyse date de 1994 (Michael Rathjen, Proof theory of reflection, Ann. Pure Appl. Logic 68 (1994), 181–224 ; disponible ici mais avec malheureusement plein de fautes typographiques pénibles). C'est l'ordinal calculable le plus grand que j'ai l'impression de bien comprendre, au sens où je crois pouvoir coder sur un ordinateur un système de notations ordinales jusqu'à ce ψ0Ω(εK+1) (ce n'est pas un exploit : l'article de Rathjen donne en gros tous les algorithmes nécessaires).

	Mais au-delà de ça, il arrive des complications bien plus importantes : pour écraser un ordinal « Π₄-réfléchissant », on doit commencer à gérer des ordinaux dont la description est vraiment plus complexe que l'écrasement de quelque chose (par exemple des ordinaux Π₂-réfléchissant sur les Π₃-réfléchissants) : les fonctions d'écrasement prennent en argument non pas juste un ordinal vers lequel écraser et un simple niveau de Mahloïtude, mais des données beaucoup plus riches que sont des « configurations de réflexion » ou des « instances de réflexion » (on n'écrase pas juste vers un ordinal de niveau de Mahloïtude ξ et inférieur à π mais vers un ordinal ayant certaines propriétés de réflexion qui conduisent elles-mêmes à d'autres fonctions d'écrasement), et le système de notation devient incroyablement plus subtil et défini par un nombre assez impressionnant de récursions imbriquées. Au moins les ordinaux « Π₅-réfléchissants » ou plus n'apportent-ils pas plus de complexité substantielle par rapport aux Π₄-réfléchissants, mais il y a encore quelques subtilités si on veut inclure tous les niveaux d'un coup, voire, des niveaux indicés par le système d'ordinaux qu'on est en train de définir. C'est en gros à ce point-là que travaille la thèse de Jan-Carl Stegert (Ordinal proof theory of Kripke-Platek set theory augmented by strong reflection principles (2010), disponible ici en PDF), qui introduit des systèmes de notations ordinales dont la seule définition s'étend sur un bon nombre de pages (notamment p. 13–30 pour le système principal, p. 68–70 pour une version simplifiée, p. 66–67 pour une version encore plus simplifiée équivalente à l'écrasement d'un cardinal Mahlo / ordinal Π₃-réfléchissant, et p. 100–113 pour un système encore plus riche). De ce que je sais, c'est le système de notations ordinales explicite le plus grand qui ait été introduit et rigoureusement analysé dans la littérature mathématique.
	
\end{changemargin}
\bigskip
 
 Another (conjectured) way to define large countable ordinals (and also large (uncountable) cardinals) is Taranovsky Ordinal Notation (TON). This formalism is much more simpler than large cardinals. 
 
 Correspondences between TON and OCF can be found in :
 
 \begin{itemize}
  \item https://googology.wikia.org/wiki/User\_blog:Boboris02/Analysis\_of\_Taranovsky%27s\_Ordinal\_Notation\_with\_%22standard\_OCFs.%22
  \item https://googology.wikia.org/wiki/User\_blog:Boboris02/Algorithm\_for\_Generating\_LUCOs\_From\_TON\_Expressions
  
  \_\%2B\_Intuitive\_Analysis
  \item https://googology.wikia.org/wiki/User\_blog:Hyp\_cos/TON,\_stable\_ordinals,\_and\_my\_array\_notation
\end{itemize}

In https://googology.wikia.org/wiki/User\_blog:Boboris02/Algorithm\_for\_Generating\_LUCOs\_From\_TON\_Expressions

\_\%2B\_Intuitive\_Analysis 

the following notations are introduced :

 $a = C(\Omega_2,2,0)$
 
 $\pi_+ = C(\Omega_2,\pi)$
 
In https://googology.wikia.org/wiki/User\_blog:Hyp\_cos/TON,\_stable\_ordinals,\_and\_my\_array\_notation 

an ordinal d is defined by :

 $d = C(\Omega_2,C(\Omega_2 2, 0))$
 
So we have $d = a_+$.

TON standard form uses only 0, C and $\Omega_n$, not arithmetic operators +, product...
For example $\Omega_2 2 = C(\Omega_2,\Omega_2)$.

Check it with Taranovsky's ordinal arithmetic program ( http://log.chez.com/src/ord/oa.py ) :

\begin{verbatim}
pi@raspberrypi:/perso/ord $ python
Python 2.7.13 (default, Sep 26 2018, 18:42:22)
[GCC 6.3.0 20170516] on linux2
Type "help", "copyright", "credits" or "license" for more information.
>>> execfile("oa.py")
>>> C2(W(2),W(2))
W(2)*2
>>>
\end{verbatim}

$C(a,b) = b+\omega^a if C(a,b) \ge a$

We have $d = \omega^d, d^2 = \omega^d \omega^d = \omega^{d 2} ; d 2 = d + d = \omega^d + d = C(d,d) ; d^2 = C(C(d,d),d)$ ( not C(C(d,d),0) because C(C(d,d),0) < d ) ; $d^d = C(C(C(d,d),d),d) ; d^{d^d} = C(C(C(C(d,d),d),d),d) ; ... $

\begin{verbatim}
pi@raspberrypi:/perso/ord $ python
Python 2.7.13 (default, Sep 26 2018, 18:42:22)
[GCC 6.3.0 20170516] on linux2
Type "help", "copyright", "credits" or "license" for more information.
>>> execfile("oa.py")

>>> d=C2(W(2),C2(W(2)*2,0))
>>> d
C(W(2)**2+W(2))

>>> d**2
C(W(2)**2+W(2))**2
>>> C2(C2(d,d),0)
C(C(W(2)**2+W(2))**2)

>>> C2(C2(d,d),d)
C(W(2)**2+W(2))**2


pi@raspberrypi:/perso/ord $ hugs ton_explore.hs
__   __ __  __  ____   ___      _________________________________________
||   || ||  || ||  || ||__      Hugs 98: Based on the Haskell 98 standard
||___|| ||__|| ||__||  __||     Copyright (c) 1994-2005
||---||         ___||           World Wide Web: http://haskell.org/hugs
||   ||                         Bugs: http://hackage.haskell.org/trac/hugs
||   || Version: September 2006 _________________________________________

Haskell 98 mode: Restart with command line option -98 to enable extensions

Type :? for help

Main> let d = C W (C (C W W) O) in C (C d d) O > d
False
Main> let d = C W (C (C W W) O) in C (C d d) d > d
True
\end{verbatim}

According to  https://googology.wikia.org/wiki/User\_blog:Hyp\_cos/TON,\_stable\_ordinals,\_and\_my\_array\_notation we have :

\begin{itemize}
 \item $I = C(\Omega_2+d,0)$
 \item $M = C(\Omega_2+d^2,0)$
 \item $K = C(\Omega_2+d^d,0)$
\end{itemize}

The standard forms are :

\begin{itemize}
 \item $\Omega_2 + d = \Omega_2 + \omega^d = C(d,\Omega_2)$
 \item $\Omega_2 + d^2 = \Omega_2 + \omega^{d 2} = C(d 2,\Omega_2) = C(C(d,d),\Omega_2)$
 \item $\Omega_2 + d^d = \Omega_2 + \omega^{d^2} = C(d^2,\Omega_2) = C(C(C(d,d),d),\Omega_2)$
\end{itemize}

\begin{verbatim}
pi@raspberrypi:/perso/ord $ python
Python 2.7.13 (default, Sep 26 2018, 18:42:22)
[GCC 6.3.0 20170516] on linux2
Type "help", "copyright", "credits" or "license" for more information.
>>> execfile('oa.py')
>>> d = C2(W(2),C2(C2(W(2),W(2)),0))
>>> d
C(W(2)**2+W(2))
>>> W(2)+d
W(2)+C(W(2)**2+W(2))
>>> C2(d,W(2))
W(2)+C(W(2)**2+W(2))
>>> C2(d*2,W(2))
W(2)+C(W(2)**2+W(2))**2
>>> C2(C2(d,d),W(2))
W(2)+C(W(2)**2+W(2))**2
>>> W(2)+d**d
W(2)+C(W(2)**2+W(2))**C(W(2)**2+W(2))
>>> C2(d**2,W(2))
W(2)+C(W(2)**2+W(2))**C(W(2)**2+W(2))
>>> C2(C2(C2(d,d),d),W(2))
W(2)+C(W(2)**2+W(2))**C(W(2)**2+W(2))
>>>
\end{verbatim}

The collapses of large cardinals can also be represented with TON.

According to  

https://googology.wikia.org/wiki/User\_blog:Boboris02/Algorithm\_for\_Generating\_LUCOs\_From\_TON\_Expressions

\_\%2B\_Intuitive\_Analysis :

$\Psi^{CK}_I(I) = C(\Omega_2+a 2, 0) = C(C(a,C(a,\Omega_2)),0)$

where $a = C(\Omega_2 2, 0) = C(C(\Omega_2,\Omega_2),0)$

and $\Omega_2+a 2 = \Omega_2+\omega^a+\omega^a = C(a,C(a,\Omega_2))$

\begin{verbatim}
pi@raspberrypi:/perso/ord $ python
Python 2.7.13 (default, Sep 26 2018, 18:42:22)
[GCC 6.3.0 20170516] on linux2
Type "help", "copyright", "credits" or "license" for more information.
>>> execfile('oa.py')
>>> a=C2(W(2)*2,0)
>>> a
C(W(2)**2)
>>> C2(C2(W(2),W(2)),0)
C(W(2)**2)
>>> W(2)+a*2
W(2)+C(W(2)**2)*2
>>> C2(a,C2(a,W(2)))
W(2)+C(W(2)**2)*2
>>>
\end{verbatim}

According to https://googology.wikia.org/wiki/User\_blog:Boboris02/Analysis\_of\_Taranovsky%27s\_Ordinal\_Notation\_with\_%22standard\_OCFs.%22 :

$\Psi(\Psi_I(0)) = C(C(\Omega_2 2,0),0) = C(C(C(\Omega_2,\Omega_2),0),0)$

and

$ \Psi(\Psi_I(I)) = C(C(\Omega_2 2+C(\Omega_2+a 2,0),0),0) $

$  = C(C(C(\Omega_2,\Omega_2)+C(C(a,C(a,\Omega_2)),0),0),0) $
 
$  = C(C(C(C(C(a,C(a,\Omega_2)),0),C(\Omega_2,\Omega_2)),0),0) $

\begin{verbatim}
pi@raspberrypi:/perso/ord $ python
Python 2.7.13 (default, Sep 26 2018, 18:42:22)
[GCC 6.3.0 20170516] on linux2
Type "help", "copyright", "credits" or "license" for more information.
>>> execfile('oa.py')
>>> a=C2(W(2)*2,0)
>>> C2(C2(W(2)*2+C2(W(2)+a*2,0),0),0)
C(C(W(2)**2*C(W(2)*C(W(2)**2)**2)))
>>> C2(C2(C2(C2(C2(a,C2(a,W(2))),0),C2(W(2),W(2))),0),0)
C(C(W(2)**2*C(W(2)*C(W(2)**2)**2)))
>>> C2(C2(W(2)*2+C2(W(2)+a*2,0),0),0)==C2(C2(C2(C2(C2(a,C2(a,W(2))),0),C2(W(2),W(2))),0),0)
True
>>>
\end{verbatim}

Here are TON graphical representations of some ordinals and cardinals previously mentioned :


\setlength{\unitlength}{1mm}
\begin{picture}(30.0,30.0)
\put(0.0,0.0){\line(0,1){6.0}}
\put(0.0,0.0){\line(1,0){6.0}}
\put(0.0,6.0){\line(0,1){4.8}}
\put(0.0,6.0){\line(1,0){4.8}}
\put(0.0,10.8){\circle*{2}}
\put(0.0,10.8){\circle{4}}
\put(4.8,6.0){\circle*{2}}
\put(4.8,6.0){\circle{4}}
\put(6.0,0.0){\circle{2}}
\end{picture}

$a = C(\Omega_2 2,0)$ = $C(C(\Omega_2,\Omega_2),0)$

\setlength{\unitlength}{1mm}
\begin{picture}(30.0,30.0)
\put(0.0,0.0){\line(0,1){6.0}}
\put(0.0,0.0){\line(1,0){6.0}}
\put(0.0,6.0){\circle*{2}}
\put(0.0,6.0){\circle{4}}
\put(6.0,0.0){\line(0,1){4.8}}
\put(6.0,0.0){\line(1,0){4.8}}
\put(6.0,4.8){\line(0,1){3.84}}
\put(6.0,4.8){\line(1,0){3.84}}
\put(6.0,8.64){\circle*{2}}
\put(6.0,8.64){\circle{4}}
\put(9.84,4.8){\circle*{2}}
\put(9.84,4.8){\circle{4}}
\put(10.8,0.0){\circle{2}}
\end{picture}

$d$ = $C(\Omega_2,C(C(\Omega_2,\Omega_2),0))$

\setlength{\unitlength}{1mm}
\begin{picture}(30.0,30.0)
\put(0.0,0.0){\line(0,1){6.0}}
\put(0.0,0.0){\line(1,0){6.0}}
\put(0.0,6.0){\line(0,1){4.8}}
\put(0.0,6.0){\line(1,0){4.8}}
\put(0.0,10.8){\line(0,1){3.84}}
\put(0.0,10.8){\line(1,0){3.84}}
\put(0.0,14.64){\circle*{2}}
\put(0.0,14.64){\circle{4}}
\put(3.84,10.8){\line(0,1){3.072}}
\put(3.84,10.8){\line(1,0){3.072}}
\put(3.84,13.872){\line(0,1){2.4576}}
\put(3.84,13.872){\line(1,0){2.4576}}
\put(3.84,16.3296){\circle*{2}}
\put(3.84,16.3296){\circle{4}}
\put(6.2976,13.872){\circle*{2}}
\put(6.2976,13.872){\circle{4}}
\put(6.912,10.8){\circle{2}}
\put(4.8,6.0){\circle*{2}}
\put(4.8,6.0){\circle{4}}
\put(6.0,0.0){\circle{2}}
\end{picture}

$I = C(\Omega_2+d,0) = C(C(d,\Omega_2),0)$ = $C(C(C(\Omega_2,C(C(\Omega_2,\Omega_2),0)),\Omega_2),0)$

\setlength{\unitlength}{1mm}
\begin{picture}(50.0,50.0)
\put(0.0,0.0){\line(0,1){10.0}}
\put(0.0,0.0){\line(1,0){10.0}}
\put(0.0,10.0){\line(0,1){8.0}}
\put(0.0,10.0){\line(1,0){8.0}}
\put(0.0,18.0){\line(0,1){6.4}}
\put(0.0,18.0){\line(1,0){6.4}}
\put(0.0,24.4){\line(0,1){5.12}}
\put(0.0,24.4){\line(1,0){5.12}}
\put(0.0,29.52){\line(0,1){4.096}}
\put(0.0,29.52){\line(1,0){4.096}}
\put(0.0,33.616){\circle*{2}}
\put(0.0,33.616){\circle{4}}
\put(4.096,29.52){\line(0,1){3.2768}}
\put(4.096,29.52){\line(1,0){3.2768}}
\put(4.096,32.7968){\line(0,1){2.62144}}
\put(4.096,32.7968){\line(1,0){2.62144}}
\put(4.096,35.41824){\circle*{2}}
\put(4.096,35.41824){\circle{4}}
\put(6.71744,32.7968){\circle*{2}}
\put(6.71744,32.7968){\circle{4}}
\put(7.3728,29.52){\circle{2}}
\put(5.12,24.4){\line(0,1){4.096}}
\put(5.12,24.4){\line(1,0){4.096}}
\put(5.12,28.496){\circle*{2}}
\put(5.12,28.496){\circle{4}}
\put(9.216,24.4){\line(0,1){3.2768}}
\put(9.216,24.4){\line(1,0){3.2768}}
\put(9.216,27.6768){\line(0,1){2.62144}}
\put(9.216,27.6768){\line(1,0){2.62144}}
\put(9.216,30.29824){\circle*{2}}
\put(9.216,30.29824){\circle{4}}
\put(11.83744,27.6768){\circle*{2}}
\put(11.83744,27.6768){\circle{4}}
\put(12.4928,24.4){\circle{2}}
\put(6.4,18.0){\circle{2}}
\put(8.0,10.0){\circle*{2}}
\put(8.0,10.0){\circle{4}}
\put(10.0,0.0){\circle{2}}
\end{picture}

$M = C(\Omega_2+d^2,0) = C(C(C(C(d,d),0),\Omega_2),0)$ = $C(C(C(C(C(\Omega_2,C(C(\Omega_2,\Omega_2),0)),C(\Omega_2,C(C(\Omega_2,\Omega_2),0))),0),\Omega_2),0)$

\setlength{\unitlength}{1mm}
\begin{picture}(50.0,50.0)
\put(0.0,0.0){\line(0,1){10.0}}
\put(0.0,0.0){\line(1,0){10.0}}
\put(0.0,10.0){\line(0,1){8.0}}
\put(0.0,10.0){\line(1,0){8.0}}
\put(0.0,18.0){\line(0,1){6.4}}
\put(0.0,18.0){\line(1,0){6.4}}
\put(0.0,24.4){\line(0,1){5.12}}
\put(0.0,24.4){\line(1,0){5.12}}
\put(0.0,29.52){\circle*{2}}
\put(0.0,29.52){\circle{4}}
\put(5.12,24.4){\circle*{2}}
\put(5.12,24.4){\circle{4}}
\put(6.4,18.0){\circle{2}}
\put(8.0,10.0){\line(0,1){6.4}}
\put(8.0,10.0){\line(1,0){6.4}}
\put(8.0,16.4){\line(0,1){5.12}}
\put(8.0,16.4){\line(1,0){5.12}}
\put(8.0,21.52){\line(0,1){4.096}}
\put(8.0,21.52){\line(1,0){4.096}}
\put(8.0,25.616){\circle*{2}}
\put(8.0,25.616){\circle{4}}
\put(12.096,21.52){\circle*{2}}
\put(12.096,21.52){\circle{4}}
\put(13.12,16.4){\circle{2}}
\put(14.4,10.0){\circle*{2}}
\put(14.4,10.0){\circle{4}}
\put(10.0,0.0){\circle{2}}
\end{picture}

$\Psi^{CK}_I(I) = C(\Omega_2+a 2) = C(C(a,C(a,\Omega_2)),0)$ = $C(C(C(C(\Omega_2,\Omega_2),0),C(C(C(\Omega_2,\Omega_2),0),\Omega_2)),0)$

\setlength{\unitlength}{1mm}
\begin{picture}(50.0,50.0)
\put(0.0,0.0){\line(0,1){10.0}}
\put(0.0,0.0){\line(1,0){10.0}}
\put(0.0,10.0){\line(0,1){8.0}}
\put(0.0,10.0){\line(1,0){8.0}}
\put(0.0,18.0){\line(0,1){6.4}}
\put(0.0,18.0){\line(1,0){6.4}}
\put(0.0,24.4){\circle*{2}}
\put(0.0,24.4){\circle{4}}
\put(6.4,18.0){\circle*{2}}
\put(6.4,18.0){\circle{4}}
\put(8.0,10.0){\circle{2}}
\put(10.0,0.0){\circle{2}}
\end{picture}

$\Psi(\Psi_I(0) = C(C(\Omega_2 2,0),0)$ = $C(C(C(\Omega_2,\Omega_2),0),0)$

\setlength{\unitlength}{1mm}
\begin{picture}(50.0,50.0)
\put(0.0,0.0){\line(0,1){10.0}}
\put(0.0,0.0){\line(1,0){10.0}}
\put(0.0,10.0){\line(0,1){8.0}}
\put(0.0,10.0){\line(1,0){8.0}}
\put(0.0,18.0){\line(0,1){6.4}}
\put(0.0,18.0){\line(1,0){6.4}}
\put(0.0,24.4){\line(0,1){5.12}}
\put(0.0,24.4){\line(1,0){5.12}}
\put(0.0,29.52){\line(0,1){4.096}}
\put(0.0,29.52){\line(1,0){4.096}}
\put(0.0,33.616){\line(0,1){3.2768}}
\put(0.0,33.616){\line(1,0){3.2768}}
\put(0.0,36.8928){\line(0,1){2.62144}}
\put(0.0,36.8928){\line(1,0){2.62144}}
\put(0.0,39.51424){\circle*{2}}
\put(0.0,39.51424){\circle{4}}
\put(2.62144,36.8928){\circle*{2}}
\put(2.62144,36.8928){\circle{4}}
\put(3.2768,33.616){\circle{2}}
\put(4.096,29.52){\line(0,1){3.2768}}
\put(4.096,29.52){\line(1,0){3.2768}}
\put(4.096,32.7968){\line(0,1){2.62144}}
\put(4.096,32.7968){\line(1,0){2.62144}}
\put(4.096,35.41824){\line(0,1){2.097152}}
\put(4.096,35.41824){\line(1,0){2.097152}}
\put(4.096,37.515392){\circle*{2}}
\put(4.096,37.515392){\circle{4}}
\put(6.193152,35.41824){\circle*{2}}
\put(6.193152,35.41824){\circle{4}}
\put(6.71744,32.7968){\circle{2}}
\put(7.3728,29.52){\circle*{2}}
\put(7.3728,29.52){\circle{4}}
\put(5.12,24.4){\circle{2}}
\put(6.4,18.0){\line(0,1){5.12}}
\put(6.4,18.0){\line(1,0){5.12}}
\put(6.4,23.12){\circle*{2}}
\put(6.4,23.12){\circle{4}}
\put(11.52,18.0){\circle*{2}}
\put(11.52,18.0){\circle{4}}
\put(8.0,10.0){\circle{2}}
\put(10.0,0.0){\circle{2}}
\end{picture}

$\Psi(\Psi_I(I))$ = $C(C(C(C(C(C(C(\Omega_2,\Omega_2),0),C(C(C(\Omega_2,\Omega_2),0),\Omega_2)),0),C(\Omega_2,\Omega_2)),0),0)$


\bigskip

Here is the Haskell program generating these graphics :

\verbatiminput{ocf_and_ton_graph.hs}

\end{document}
