\documentclass[10pt]{article}
\begin{document}

More information about this function can be fount at :

http://googology.wikia.com/wiki/Veblen\_function 

\bigskip

Every non-zero ordinal \(\alpha<\Gamma_0\), where \(\Gamma_0\) is the smallest ordinal \(\alpha\) such that \(\varphi_\alpha(0)=\alpha\), can be uniquely written in normal form for the Veblen hierarchy:

\(\alpha=\varphi_{\beta_1}(\gamma_1) + \varphi_{\beta_2}(\gamma_2) + \cdots + \varphi_{\beta_k}(\gamma_k)\),

where

\(\varphi_{\beta_1}(\gamma_1) \ge \varphi_{\beta_2}(\gamma_2) \ge \cdots \ge \varphi_{\beta_k}(\gamma_k)\)
\(\gamma_m < \varphi_{\beta_m}(\gamma_m)\) for \(m \in \{1,...,k\}\)

\bigskip

Now we will see how we can find the fundamental sequence of an ordinal written in this notmal form.

From the rule defining addition of a limit ordinal :

\( \alpha + lim(f) = lim (n \mapsto \alpha + f(n)) \)

we deduce the fundamental sequence :

\( (\alpha + \beta)[n] = \alpha + \beta[n] \)

if \( \beta \) is a limit ordinal. 

In particular, we have :

\((\varphi_{\beta_1}(\gamma_1) + \varphi_{\beta_2}(\gamma_2) + \cdots + \varphi_{\beta_k}(\gamma_k))[n]=\varphi_{\beta_1}(\gamma_1) + \cdots + \varphi_{\beta_{k-1}}(\gamma_{k-1}) + (\varphi_{\beta_k}(\gamma_k) [n])\), where \(\varphi_{\beta_1}(\gamma_1) \ge \varphi_{\beta_2}(\gamma_2) \ge \cdots \ge \varphi_{\beta_k}(\gamma_k)\) and \(\gamma_m < \varphi_{\beta_m}(\gamma_m)\) for \(m \in \{1,2,...,k\}\),

Then, \( \varphi_0(\gamma) \) is \( \omega^\gamma \).

For \( \gamma = 0 \) it is 1.

From the rule of multiplication by a limit ordinal :

\( \alpha \cdot lim(f) = lim (n \mapsto \alpha \cdot f(n)) \)

we deduce the fundamental sequence :

\( (\alpha \cdot \beta)[n] = \alpha \cdot \beta[n] \)
if \( \beta \) is a limit ordinal.

In particular, for \( \omega \) :

\( (\alpha \cdot \omega)[n] = \alpha \cdot \omega[n] = \alpha \cdot n \)

Then we have :

\( \varphi_0(\gamma+1) = \omega^{\gamma+1} = \omega^\gamma \cdot \omega = \varphi_0(\gamma) \cdot \omega \)

So the corresponding fundamental sequence is : 

\( \varphi_0(\gamma+1)[n] = (\varphi_0(\gamma) \cdot \omega)[n] = \varphi_0(\gamma) \cdot n \)

If \( \gamma \) is a limit ordinal, the fundamental sequence can be defined canonically:

\( \varphi_0(\gamma)[n] = \varphi_0(\gamma[n]) \)

Then, \( \varphi_{\beta+1}(\gamma) \) is the \(1+\gamma\)-th fixed point of the function \( \xi \mapsto \varphi_\beta(\xi) \), or more simply the function \( \varphi_\beta \).

In particular, \( \varphi_{\beta+1}(0) \) is the least fixed point of this function, which is \( {\varphi_\beta}^\omega(0) \). A fundamental sequence of this ordinal is \( \varphi_{\beta+1}(0)[n] = {\varphi_\beta}^n(0) \), which can also be written \( \varphi_{\beta+1}(0)[0] = 0 \) and \( \varphi_{beta+1}(0)[n+1] = \varphi_\beta(\varphi_{\beta+1}(0)[n]) \).

\( \varphi_{\beta+1}(\gamma+1) \) is the fixed point of \( \varphi_\beta \) that follows \( \varphi_{\beta+1}(\gamma) \). It is \( {\varphi_\beta}^\omega(\varphi_{\beta+1}(\gamma)+1) \). This can also be written \( \varphi_{\beta+1}(\gamma+1)[0] = \varphi_{\beta+1}(\gamma)+1 \) and \( \varphi_{\beta+1}(\gamma+1)[n+1] = \varphi_\beta(\varphi_{\beta+1}(\gamma+1)[n]) \).

If \( \gamma \) is a limit ordinal, the fundamental sequence can be defined canonically:

\( \varphi_{\beta+1}(\gamma)[n] = \varphi_{\beta+1}(\gamma[n]) \).

Finally, if \( \beta \) is a limit ordinal, we can define canonically:

\( \varphi_\beta(0)[n] =  \varphi_{\beta[n]}(0) \) if \( \beta < \varphi_\beta(0) \)

and

\( \varphi_\beta(\gamma+1)[n] = \varphi_{\beta[n]}(\varphi_\beta(\gamma)+1) \)

Note that if we remove the condition \( \beta < \varphi_\beta(0) \) there is a problem. For example, if we take \( \beta = \Gamma_0 \) the least fixed point of the function \( \xi \mapsto \varphi_\xi(0) \), then we have \( \varphi_{\Gamma_0}(0) = \Gamma_0 \).
A fundamental sequence of \( \Gamma_0 \) is \( \Gamma_0[0] = 0, \Gamma_0[1] = \varphi_0(0) = \omega^0 = 1, \Gamma_0[2] = \varphi_1(0) = \varepsilon_0, \ldots \).
 Then we have \( \varphi_{\Gamma_0}(0)[0] = \Gamma_0[0] = 0 \), but \( \varphi_{\Gamma_0[0]}(0) = \varphi_0(0) = \omega^0 = 1 \).

\bigskip

Let us recap now the results we obtained.

The fundamental sequences for the Veblen functions \(\varphi_\beta(\gamma) = \varphi(\beta,\gamma) \) are :

\bigskip

(1) \((\varphi_{\beta_1}(\gamma_1) + \varphi_{\beta_2}(\gamma_2) + \cdots + \varphi_{\beta_k}(\gamma_k))[n]=\varphi_{\beta_1}(\gamma_1) + \cdots + \varphi_{\beta_{k-1}}(\gamma_{k-1}) + (\varphi_{\beta_k}(\gamma_k) [n])\), where \(\varphi_{\beta_1}(\gamma_1) \ge \varphi_{\beta_2}(\gamma_2) \ge \cdots \ge \varphi_{\beta_k}(\gamma_k)\) and \(\gamma_m < \varphi_{\beta_m}(\gamma_m)\) for \(m \in \{1,2,...,k\}\),

(2) \(\varphi_0(0) = 1\),

(3) \(\varphi_0(\gamma+1) [n] = \varphi_0(\gamma) n\)

(4) \(\varphi_{\beta+1}(0) [0] = 0 \) and \(\varphi_{\beta+1}(0) [n+1] = \varphi_{\beta}(\varphi_{\beta+1}(0) [n]) \),

(5) \(\varphi_{\beta+1}(\gamma+1) [0] = \varphi_{\beta+1}(\gamma)+1 \,\) and \(\varphi_{\beta+1}(\gamma+1) [n+1] = \varphi_{\beta} (\varphi_{\beta+1}(\gamma+1) [n]) \),

(6) \(\varphi_{\beta}(\gamma) [n] = \varphi_{\beta}(\gamma [n])\) for a limit ordinal \(\gamma<\varphi_\beta(\gamma)\),

(7) \(\varphi_{\beta}(0) [n] = \varphi_{\beta [n]}(0)\) for a limit ordinal \(\beta<\varphi_\beta(0)\),

(8) \(\varphi_{\beta}(\gamma+1) [n] = \varphi_{\beta [n]}(\varphi_{\beta}(\gamma)+1)\) for a limit ordinal \(\beta\).

\bigskip

From these fundamental sequences, we can retrieve the initial definition of the function \( \varphi \).

\bigskip

(1) This does not concern the definition of the \( \varphi \) function but the definition of addition

(2) and (3) and (6) for \( \beta = 0 \) are equivalent to \( \varphi_0(\gamma) = \omega^\gamma \).

(4) \( \varphi_{\beta+1}(0) = lim (n \mapsto {\varphi_\beta}^n(0)) = {\varphi_\beta}^\omega(0) \) which is the least fixed point of \( \varphi_\beta \).

(5) \( \varphi_{\beta+1}(\gamma+1) = lim (n \mapsto {\varphi_\beta}^n (\varphi_{\beta+1}(\gamma)+1)) \), which is the least fixed point of \( \varphi_\beta \) strictly greater than \( \varphi_{\beta+1}(\gamma) \), so with (6) it gives \( \varphi_{\beta+1}(\gamma) \) is the \(1+\gamma\)-th fixed point of \( \varphi_\beta \).

(7), (8) and (6) for \( \beta \) limit ordinal complete the definition of \( \varphi_\beta(\gamma) \) for \( \beta \) limit ordinal.

\bigskip

Here is an Haskell implementation of the \( \varphi \) function : 

\begin{verbatim}

module Phi where
 
 data Nat 
  = ZeroN 
  | SucN Nat

 data Ord 
  = Zero 
  | Suc Ord 
  | Lim (Nat -> Ord)

 iter f ZeroN x = x
 iter f (SucN n) x = f (iter f n x)

 opLim f a = Lim (\n -> f n a)

 opItw f = opLim (iter f)

 cantor a Zero = Suc a
 cantor a (Suc b) = opItw (\x -> cantor x b) a
 cantor a (Lim f) = Lim (\n -> cantor a (f n))

 nabla f Zero = f Zero
 nabla f (Suc a) = f (Suc (nabla f a))
 nabla f (Lim h) = Lim (\n -> nabla f (h n))

 deriv f = nabla (opItw f)

 phi Zero = cantor Zero
 phi (Suc a) = deriv (phi a)
 phi (Lim f) = nabla (opLim (\n -> phi (f n)))

\end{verbatim}

iter f n x is \( f^n(x) \).

opLim f a is the limit of f 0 a, f 1 a, f 2 a, ...

opItw f is \( f^\omega \)

cantor a b is \( a + \omega^b \).

deriv f a is the (1+a)-th fixed point of f.

\end{document}

