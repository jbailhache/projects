\documentclass[10pt]{article}
\begin{document}

More information about this function can be fount at :

http://googology.wikia.com/wiki/Veblen\_function 

\bigskip

Every non-zero ordinal \(\alpha<\Gamma_0\), where \(\Gamma_0\) is the smallest ordinal \(\alpha\) such that \(\varphi_\alpha(0)=\alpha\), can be uniquely written in normal form for the Veblen hierarchy:

\(\alpha=\varphi_{\beta_1}(\gamma_1) + \varphi_{\beta_2}(\gamma_2) + \cdots + \varphi_{\beta_k}(\gamma_k)\),

where

\(\varphi_{\beta_1}(\gamma_1) \ge \varphi_{\beta_2}(\gamma_2) \ge \cdots \ge \varphi_{\beta_k}(\gamma_k)\)
\(\gamma_m < \varphi_{\beta_m}(\gamma_m)\) for \(m \in \{1,...,k\}\)

\bigskip

Now we will see how we can find the fundamental sequence of an ordinal written in this notmal form.

From the rule defining addition of a limit ordinal :
\[ \alpha + lim(f) = lim (n \mapsto \alpha + f(n)) \]
we deduce the fundamental sequence :
\[ (\alpha + \beta)[n] = \alpha + \beta[n] \]
if \( \beta \) is a limit ordinal. 

In particular, we have :

\((\varphi_{\beta_1}(\gamma_1) + \varphi_{\beta_2}(\gamma_2) + \cdots + \varphi_{\beta_k}(\gamma_k))[n]=\varphi_{\beta_1}(\gamma_1) + \cdots + \varphi_{\beta_{k-1}}(\gamma_{k-1}) + (\varphi_{\beta_k}(\gamma_k) [n])\), where \(\varphi_{\beta_1}(\gamma_1) \ge \varphi_{\beta_2}(\gamma_2) \ge \cdots \ge \varphi_{\beta_k}(\gamma_k)\) and \(\gamma_m < \varphi_{\beta_m}(\gamma_m)\) for \(m \in \{1,2,...,k\}\),

Then, \( \varphi_0(\gamma) \) is \( \omega^\gamma \).

For \( \gamma = 0 \) it is 1.

From the rule of multiplication by a limit ordinal :

\( \alpha \cdot lim(f) = lim (n \mapsto \alpha \cdot f(n)) \)

we deduce the fundamental sequence :

\( (\alpha \cdot \beta)[n] = \alpha \cdot \beta[n] \)
if \( \beta \) is a limit ordinal.

In particular, for \( \omega \) :

\( (\alpha \cdot \omega)[n] = \alpha \cdot \omega[n] = \alpha \cdot n \)

Then we have :

\( \varphi_0(\gamma+1) = \omega^{\gamma+1} = \omega^\gamma \cdot \omega = \varphi_0(\gamma) \cdot \omega \)

So the corresponding fundamental sequence is : 

\( \varphi_0(\gamma+1)[n] = (\varphi_0(\gamma) \cdot \omega)[n] = \varphi_0(\gamma) \cdot n \)

If \( \gamma \) is a limit ordinal, the fundamental sequence is defined canonically :

\( \varphi_0(\gamma)[n] = \varphi_0(\gamma[n]) \)



\bigskip

This function can be defined with fundamental sequences.

The fundamental sequences for the Veblen functions \(\varphi_\beta(\gamma) = \varphi(\beta,\gamma) \) are :

\begin{enumerate}
     \setlength{\itemsep}{1pt}
     \setlength{\parskip}{0pt}
     \setlength{\parsep}{0pt}

\item \((\varphi_{\beta_1}(\gamma_1) + \varphi_{\beta_2}(\gamma_2) + \cdots + \varphi_{\beta_k}(\gamma_k))[n]=\varphi_{\beta_1}(\gamma_1) + \cdots + \varphi_{\beta_{k-1}}(\gamma_{k-1}) + (\varphi_{\beta_k}(\gamma_k) [n])\), where \(\varphi_{\beta_1}(\gamma_1) \ge \varphi_{\beta_2}(\gamma_2) \ge \cdots \ge \varphi_{\beta_k}(\gamma_k)\) and \(\gamma_m < \varphi_{\beta_m}(\gamma_m)\) for \(m \in \{1,2,...,k\}\),

\item \(\varphi_0(0) = 1\),

\item \(\varphi_0(\gamma+1) [n] = \varphi_0(\gamma) n\)

\item \(\varphi_{\beta+1}(0) [0] = 0 \) and \(\varphi_{\beta+1}(0) [n+1] = \varphi_{\beta}(\varphi_{\beta+1}(0) [n]) \),

\item \(\varphi_{\beta+1}(\gamma+1) [0] = \varphi_{\beta+1}(\gamma)+1 \,\) and \(\varphi_{\beta+1}(\gamma+1) [n+1] = \varphi_{\beta} (\varphi_{\beta+1}(\gamma+1) [n]) \),

\item \(\varphi_{\beta}(\gamma) [n] = \varphi_{\beta}(\gamma [n])\) for a limit ordinal \(\gamma<\varphi_\beta(\gamma)\),

\item \(\varphi_{\beta}(0) [n] = \varphi_{\beta [n]}(0)\) for a limit ordinal \(\beta<\varphi_\beta(0)\),

\item \(\varphi_{\beta}(\gamma+1) [n] = \varphi_{\beta [n]}(\varphi_{\beta}(\gamma)+1)\) for a limit ordinal \(\beta\).

\end{enumerate}

These fundamental sequences can be reformulated to get a definition of the function \( \varphi \).

\begin{enumerate}
     \setlength{\itemsep}{1pt}
     \setlength{\parskip}{0pt}
     \setlength{\parsep}{0pt}

\item This does not concern the definition of the \( \varphi \) function but the definition of addition

\item and

\item are equivalent to \( \varphi_0(\gamma) = \omega^\gamma \).

\item \( \varphi_{\beta+1}(0) = lim (n \mapsto {\varphi_\beta}^n(0)) = {\varphi_\beta}^\omega(0) \) which is the least fixed point of \( \varphi_\beta \).

\item \( \varphi_{\beta+1}(\gamma+1) = lim (n \mapsto {\varphi_\beta}^n (\varphi_{\beta+1}(\gamma)+1)) \), which is the least fixed point of \( \varphi_\beta \) strictly greater than \( \varphi_{\beta+1}(\gamma) \), so \( \varphi_{\beta+1}(\gamma) \) is the \(1+\gamma\)-th fixed point of \( \varphi_\beta \).

\end{enumerate}


\end{document}

