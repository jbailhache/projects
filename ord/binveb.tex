\documentclass[10pt]{article}
\begin{document}

This function can be defined with fundamental sequences.

The fundamental sequences for the Veblen functions \(\varphi_\beta(\gamma) = \varphi(\beta,\gamma) \) are :

\begin{enumerate}
     \setlength{\itemsep}{1pt}
     \setlength{\parskip}{0pt}
     \setlength{\parsep}{0pt}

\item \((\varphi_{\beta_1}(\gamma_1) + \varphi_{\beta_2}(\gamma_2) + \cdots + \varphi_{\beta_k}(\gamma_k))[n]=\varphi_{\beta_1}(\gamma_1) + \cdots + \varphi_{\beta_{k-1}}(\gamma_{k-1}) + (\varphi_{\beta_k}(\gamma_k) [n])\), where \(\varphi_{\beta_1}(\gamma_1) \ge \varphi_{\beta_2}(\gamma_2) \ge \cdots \ge \varphi_{\beta_k}(\gamma_k)\) and \(\gamma_m < \varphi_{\beta_m}(\gamma_m)\) for \(m \in \{1,2,...,k\}\),

\item \(\varphi_0(0) = 1\),

\item \(\varphi_0(\gamma+1) [n] = \varphi_0(\gamma) n\)

\item \(\varphi_{\beta+1}(0) [0] = 0 \) and \(\varphi_{\beta+1}(0) [n+1] = \varphi_{\beta}(\varphi_{\beta+1}(0) [n]) \),

\item \(\varphi_{\beta+1}(\gamma+1) [0] = \varphi_{\beta+1}(\gamma)+1 \,\) and \(\varphi_{\beta+1}(\gamma+1) [n+1] = \varphi_{\beta} (\varphi_{\beta+1}(\gamma+1) [n]) \),

\item \(\varphi_{\beta}(\gamma) [n] = \varphi_{\beta}(\gamma [n])\) for a limit ordinal \(\gamma<\varphi_\beta(\gamma)\),

\item \(\varphi_{\beta}(0) [n] = \varphi_{\beta [n]}(0)\) for a limit ordinal \(\beta<\varphi_\beta(0)\),

\item \(\varphi_{\beta}(\gamma+1) [n] = \varphi_{\beta [n]}(\varphi_{\beta}(\gamma)+1)\) for a limit ordinal \(\beta\).

\end{enumerate}

These fundamental sequences can be reformulated to get a definition of the function \( \varphi \).

\begin{enumerate}
     \setlength{\itemsep}{1pt}
     \setlength{\parskip}{0pt}
     \setlength{\parsep}{0pt}

\item This does not concern the definition of the \( \varphi \) function but the definition of addition

\item and

\item are equivalent to \( \varphi_0(\gamma) = \omega^\gamma \).

\item \( \varphi_{\beta+1}(0) = lim (n \mapsto {\varphi_\beta}^n(0)) = {\varphi_\beta}^\omega(0) \) which is the least fixed point of \( \varphi_\beta \).

\item \( \varphi_{\beta+1}(\gamma+1) = lim (n \mapsto {\varphi_\beta}^n (\varphi_{\beta+1}(\gamma)+1)) \), which is the least fixed point of \( \varphi_\beta \) strictly greater than \( \varphi_{\beta+1}(\gamma) \), so \( \varphi_{\beta+1}(\gamma) \) is the \(1+\gamma\)-th fixed point of \( \varphi_\beta \).

\end{enumerate}


\end{document}

