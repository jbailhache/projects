\documentclass[10pt]{article}

\usepackage[left=1cm,right=1cm,top=2cm,bottom=2cm]{geometry}
\usepackage[utf8]{inputenc}
\usepackage{amsfonts}
\usepackage{amssymb}
\usepackage{amsmath}
\usepackage{comment}

\begin{document}

\title{A Tutorial Overview of Ordinal Notations}
\author{Jacques Bailhache (jacques.bailhache@gmail.com)}

\maketitle

\setlength{\parindent}{0pt}

\section{Ordinal collapsing functions}

 Fundamental sequences for the functions collapsing weakly inaccessible cardinals 

Definition

\(\Omega_\alpha\) with \(\alpha>0\) is the \(\alpha\)-th uncountable cardinal, \(I_\alpha\) with \(\alpha>0\) is the \(\alpha\)-th weakly inaccessible cardinal and for this notation \(I_0=\Omega_0=0\).

In this section the variables \(\rho\), \(\pi\) are reserved for uncountable regular cardinals of the form \(\Omega_{\nu+1}\) or \(I_{\mu+1}\).

Then,

\(C_0(\alpha,\beta) = \beta\cup\{0\}\)
\(C_{n+1}(\alpha,\beta) = \{\gamma+\delta|\gamma,\delta\in C_n(\alpha,\beta)\}\)
\(\cup \{\Omega_\gamma|\gamma\in C_n(\alpha,\beta)\}\)
\(\cup \{I_\gamma|\gamma\in C_n(\alpha,\beta)\}\)
\(\cup \{\psi_\pi(\gamma)|\pi,\gamma\in C_n(\alpha,\beta)\wedge\gamma<\alpha\}\)
\(C(\alpha,\beta) = \bigcup_{n<\omega} C_n(\alpha,\beta)\)
\(\psi_\pi(\alpha) = \min\{\beta<\pi|C(\alpha,\beta)\cap\pi\subseteq\beta\}\)

Properties

\(\psi_{\pi}(0)=1\)
\(\psi_{\Omega_1}(\alpha)=\omega^\alpha\) for \(\alpha<\varepsilon_0\)
\(\psi_{\Omega_{\nu+1}}(\alpha)=\omega^{\Omega_\nu+\alpha}\) for \(1\le\alpha<\varepsilon_{\Omega_\nu+1}\) and \(\nu>0\)

Standard form for ordinals \(\alpha<\beta=\text{min}\{\xi|I_\xi=\xi\}\)

 The standard form for 0 is 0
 If \(\alpha\) is of the form \(\Omega_\beta\), then the standard form for \(\alpha\) is \(\alpha= \Omega_\beta\) where \(\beta<\alpha\) and \(\beta\) is expressed in standard form
 If \(\alpha\) is of the form \(I_\beta\), then the standard form for \(\alpha\) is \(\alpha= I_\beta\) where \(\beta<\alpha\) and \(\beta\) is expressed in standard form
 If \(\alpha\) is not additively principal and \(\alpha>0\), then the standard form for \(\alpha\) is \(\alpha=\alpha_1+\alpha_2+\cdots+\alpha_n\), where the \(\alpha_i\) are principal ordinals with \(\alpha_1\geq\alpha_2\geq\cdots\geq\alpha_n\), and the \(\alpha_i\) are expressed in standard form
 If \(\alpha\) is an additively principal ordinal but not of the form \(\Omega_\beta\) or \(I_\gamma\), then \(\alpha\) is expressible in the form \(\psi_\pi(\delta)\). Then the standard form for \(\alpha\) is \(\alpha=\psi_\pi(\delta)\) where \(\pi\) and \(\delta\) are expressed in standard form

Fundamental sequences

The fundamental sequence for an ordinal number \(\alpha\) with cofinality \(\text{cof}(\alpha)=\beta\) is a strictly increasing sequence \((\alpha[\eta])_{\eta<\beta}\) with length \(\beta\) and with limit \(\alpha\), where \(\alpha[\eta]\) is the \(\eta\)-th element of this sequence.

Let \(S=\{\alpha|\text{cof}(\alpha)=1\}\) and \(L=\{\alpha|\text{cof}(\alpha)\geq\omega\}\) where \(S\) denotes the set of successor ordinals and \(L\) denotes the set of limit ordinals.

For non-zero ordinals written in standard form fundamental sequences defined as follows:

If \(\alpha=\alpha_1+\alpha_2+\cdots+\alpha_n\) with \(n\geq 2\) then \(\text{cof}(\alpha)=\text{cof}(\alpha_n)\) and \(\alpha[\eta]=\alpha_1+\alpha_2+\cdots+(\alpha_n[\eta])\)
If \(\alpha=\psi_{\pi}(0)\) then \(\alpha=\text{cof}(\alpha)=1\) and \(\alpha[0]=0\)
If \(\alpha=\psi_{\Omega_{\nu+1}}(1)\) then \(\text{cof}(\alpha)=\omega\) and \(\left\{\begin{array}{lcr} \alpha[\eta]=\Omega_{\nu}\cdot\eta \text{ if }\nu>0 \\ \alpha[\eta]=\eta \text{ if }\nu=0\\ \end{array}\right.\)
If \(\alpha=\psi_{\Omega_{\nu+1}}(\beta+1)\) and \(\beta\geq 1\) then \(\text{cof}(\alpha)=\omega\) and \(\alpha[\eta]=\psi_{\Omega_{\nu+1}}(\beta)\cdot\eta\)
If \(\alpha=\psi_{ I_{\nu+1}}(1)\) then \(\text{cof}(\alpha)=\omega\) and \(\alpha[0]=I_\nu+1\) and \(\alpha[\eta+1]=\Omega_{\alpha[\eta]}\)
If \(\alpha=\psi_{ I_{\nu+1}}(\beta+1)\) and \(\beta\geq 1\) then \(\text{cof}(\alpha)=\omega\) and \(\alpha[0]=\psi_{ I_{\nu+1}}(\beta)+1\) and \(\alpha[\eta+1]=\Omega_{\alpha[\eta]}\)
If \(\alpha=\pi\) then \(\text{cof}(\alpha)=\pi\) and \(\alpha[\eta]=\eta\)
If \(\alpha=\Omega_\nu\) and \(\nu\in L\) then \(\text{cof}(\alpha)=\text{cof}(\nu)\) and \(\alpha[\eta]=\Omega_{\nu[\eta]}\)
If \(\alpha=I_\nu\) and \(\nu\in L\) then \(\text{cof}(\alpha)=\text{cof}(\nu)\) and \(\alpha[\eta]=I_{\nu[\eta]}\)
 If \(\alpha=\psi_\pi(\beta)\) and \(\omega\le\text{cof}(\beta)<\pi\) then \(\text{cof}(\alpha)=\text{cof}(\beta)\) and \(\alpha[\eta]=\psi_\pi(\beta[\eta])\)
 If \(\alpha=\psi_\pi(\beta)\) and \(\text{cof}(\beta)=\rho\geq\pi\) then \(\text{cof}(\alpha)=\omega\) and \(\alpha[\eta]=\psi_\pi(\beta[\gamma[\eta]])\) with \(\gamma[0]=1\) and \(\gamma[\eta+1]=\psi_{\rho}(\beta[\gamma[\eta]])\)

Limit of this notation is \(\lambda\). If \(\alpha=\lambda\) then \(\text{cof}(\alpha)=\omega\) and \(\alpha[0]=1\) and \(\alpha[\eta+1]=I_{\alpha[\eta]}\).

 Fundamental sequences for the functions collapsing \(\alpha\)-weakly inaccessible cardinals 

Definition

An ordinal is \(\alpha\)-weakly inaccessible if it's an uncountable regular cardinal and it's a limit of \(\gamma\)-weakly inaccessible cardinals for all \(\gamma<\alpha\).

Let \(I(\alpha,\beta)\) be the \((1+\beta)\)th \(\alpha\)-weakly inaccessible cardinal if \(\beta=0\) or \(\beta=\gamma+1\), and \(I(\alpha,\beta)=\text{sup}\{I(\alpha, \xi)|\xi<\beta\}\) if \(\beta\) is a limit ordinal.

In this section the variables \(\rho\), \(\pi\) are reserved for uncountable regular cardinals of the form \(I(\alpha,0)\) or \(I(\alpha,\beta+1)\).

Then,

\(C_0(\alpha,\beta) = \beta\cup\{0\}\)
\(C_{n+1}(\alpha,\beta) = \{\gamma+\delta|\gamma,\delta\in C_n(\alpha,\beta)\}\)
\(\cup \{I(\gamma,\delta)|\gamma,\delta\in C_n(\alpha,\beta)\}\)
\(\cup \{\psi_\pi(\gamma)|\pi,\gamma\in C_n(\alpha,\beta)\wedge\gamma<\alpha\}\)
\(C(\alpha,\beta) = \bigcup_{n<\omega} C_n(\alpha,\beta)\)
\(\psi_\pi(\alpha) = \min\{\beta<\pi|C(\alpha,\beta)\cap\pi\subseteq\beta\}\)

Properties

\(I(0,\alpha)=\Omega_{1+\alpha}=\aleph_{1+\alpha}\)
\(I(1,\alpha)=I_{1+\alpha}\)
\(\psi_{I(0,0)}(\alpha)=\omega^\alpha\) for \(\alpha<\varepsilon_0\)
\(\psi_{I(0,\alpha+1)}(\beta)=\omega^{I(0,\alpha)+1+\beta}\) for \(\beta<\varepsilon_{I(0,\alpha)+1}\)

Standard form for ordinals \(\alpha<\psi_{I(1,0,0)}(0)=\text{min}\{\xi|I(\xi,0)=\xi\}\)

 The standard form for 0 is 0
 If \(\alpha\) is of the form \(I(\beta,\gamma)\), then the standard form for \(\alpha\) is \(\alpha=I(\beta,\gamma)\) where \(\beta,\gamma<\alpha\) and \(\beta,\gamma\) are expressed in standard form
 If \(\alpha\) is not additively principal and \(\alpha>0\), then the standard form for \(\alpha\) is \(\alpha=\alpha_1+\alpha_2+\cdots+\alpha_n\), where the \(\alpha_i\) are principal ordinals with \(\alpha_1\geq\alpha_2\geq\cdots\geq\alpha_n\), and the \(\alpha_i\) are expressed in standard form
 If \(\alpha\) is an additively principal ordinal but not of the form \(I(\beta,\gamma)\), then \(\alpha\) is expressible in the form \(\psi_\pi(\delta)\). Then the standard form for \(\alpha\) is \(\alpha=\psi_\pi(\delta)\) where \(\pi\) and \(\delta\) are expressed in standard form

Fundamental sequences

The fundamental sequence for an ordinal number \(\alpha\) with cofinality \(\text{cof}(\alpha)=\beta\) is a strictly increasing sequence \((\alpha[\eta])_{\eta<\beta}\) with length \(\beta\) and with limit \(\alpha\), where \(\alpha[\eta]\) is the \(\eta\)-th element of this sequence.

Let \(S=\{\alpha|\text{cof}(\alpha)=1\}\) and \(L=\{\alpha|\text{cof}(\alpha)\geq\omega\}\) where \(S\) denotes the set of successor ordinals and \(L\) denotes the set of limit ordinals.

For non-zero ordinals \(\alpha<\psi_{I(1,0,0)}(0)\) written in standard form fundamental sequences defined as follows:[2]

If \(\alpha=\alpha_1+\alpha_2+\cdots+\alpha_n\) with \(n\geq 2\) then \(\text{cof}(\alpha)=\text{cof}(\alpha_n)\) and \(\alpha[\eta]=\alpha_1+\alpha_2+\cdots+(\alpha_n[\eta])\)
If \(\alpha=\psi_{I(0,0)}(0)\) then \(\alpha=\text{cof}(\alpha)=1\) and \(\alpha[0]=0\)
If \(\alpha=\psi_{I(0,\beta+1)}(0)\) then \(\text{cof}(\alpha)=\omega\) and \(\alpha[\eta]=I(0,\beta)\cdot\eta\)
If \(\alpha=\psi_{I(0,\beta)}(\gamma+1)\) and \(\beta\in\{0\}\cup S\) then \(\text{cof}(\alpha)=\omega\) and \(\alpha[\eta]=\psi_{I(0,\beta)}(\gamma)\cdot\eta\)
If \(\alpha=\psi_{I(\beta+1,0)}(0)\) then \(\text{cof}(\alpha)=\omega\) and \(\alpha[0]=0\) and \(\alpha[\eta+1]=I(\beta,\alpha[\eta])\)
If \(\alpha=\psi_{I(\beta+1,\gamma+1)}(0)\) then \(\text{cof}(\alpha)=\omega\) and \(\alpha[0]=I(\beta+1,\gamma)+1\) and \(\alpha[\eta+1]=I(\beta,\alpha[\eta])\)
If \(\alpha=\psi_{I(\beta+1,\gamma)}(\delta+1)\) and \(\gamma\in\{0\}\cup S\) then \(\text{cof}(\alpha)=\omega\) and \(\alpha[0]=\psi_{I(\beta+1,\gamma)}(\delta)+1\) and \(\alpha[\eta+1]=I(\beta,\alpha[\eta])\)
if \(\alpha=\psi_{I(\beta,0)}(0)\) and \(\beta\in L\) then \(\text{cof}(\alpha)=\text{cof}(\beta)\) and \(\alpha[\eta]=I(\beta[\eta],0)\)
if \(\alpha=\psi_{I(\beta,\gamma+1)}(0)\) and \(\beta\in L\) then \(\text{cof}(\alpha)=\text{cof}(\beta)\) and \(\alpha[\eta]=I(\beta[\eta],I(\beta,\gamma)+1)\)
if \(\alpha=\psi_{I(\beta,\gamma)}(\delta+1)\) and \(\beta\in L\) and \(\gamma\in \{0\}\cup S\) then \(\text{cof}(\alpha)=\text{cof}(\beta)\) and \(\alpha[\eta]=I(\beta[\eta],\psi_{I(\beta,\gamma)}(\delta)+1)\)
If \(\alpha=\pi\) then \(\text{cof}(\alpha)=\pi\) and \(\alpha[\eta]=\eta\)
If \(\alpha=I(\beta,\gamma)\) and \(\gamma\in L\) then \(\text{cof}(\alpha)=\text{cof}(\gamma)\) and \(\alpha[\eta]=I(\beta,\gamma[\eta])\)
 If \(\alpha=\psi_\pi(\beta)\) and \(\omega\le\text{cof}(\beta)<\pi\) then \(\text{cof}(\alpha)=\text{cof}(\beta)\) and \(\alpha[\eta]=\psi_\pi(\beta[\eta])\)
 If \(\alpha=\psi_\pi(\beta)\) and \(\text{cof}(\beta)=\rho\geq\pi\) then \(\text{cof}(\alpha)=\omega\) and \(\alpha[\eta]=\psi_\pi(\beta[\gamma[\eta]])\) with \(\gamma[0]=1\) and \(\gamma[\eta+1]=\psi_{\rho}(\beta[\gamma[\eta]])\)

Limit of this notation \(\psi_{I(1,0,0)}(0)\). If \(\alpha=\psi_{I(1,0,0)}(0)\) then \(\text{cof}(\alpha)=\omega\) and \(\alpha[0]=0\) and \(\alpha[\eta+1]=I(\alpha[\eta],0)\)

 The functions collapsing weakly Mahlo cardinals 

Definition

An ordinal is weakly Mahlo if it's an uncountable regular cardinal, and regular cardinals in it (in another word, less than it) are stationary.

Let \(M_0=0\), \(M_{\alpha+1}\) be the next weakly Mahlo cardinal after \(M_\alpha\), and \(M_\alpha=\sup\{M_\beta|\beta<\alpha\}\) for limit ordinal \(\alpha\). Then, \begin{eqnarray*} C_0(\alpha,\beta) &=& \beta\cup\{0\} \\ C_{n+1}(\alpha,\beta) &=& \{\gamma+\delta|\gamma,\delta\in C_n(\alpha,\beta)\} \\ &\cup& \{M_\gamma|\gamma\in C_n(\alpha,\beta)\} \\ &\cup& \{\chi_\pi(\gamma)|\pi,\gamma\in C_n(\alpha,\beta)\wedge\gamma<\alpha\wedge\pi\text{ is weakly Mahlo}\} \\ &\cup& \{\psi_\pi(\gamma)|\pi,\gamma\in C_n(\alpha,\beta)\wedge\gamma<\alpha\wedge\pi\text{ is uncountable regular}\} \\ C(\alpha,\beta) &=& \bigcup_{n<\omega} C_n(\alpha,\beta) \\ \chi_\pi(\alpha) &=& \min\{\beta<\pi|C(\alpha,\beta)\cap\pi\subseteq\beta\wedge\beta\text{ is uncountable regular}\} \\ \psi_\pi(\alpha) &=& \min\{\beta<\pi|C(\alpha,\beta)\cap\pi\subseteq\beta\} \end{eqnarray*}

In this section the variables \(\rho, \pi\) are reserved for uncountable regular cardinals of the form \(\chi_\alpha(\beta)\) or \(M_{\gamma+1}\).

Standard form for ordinals \(\alpha<\text{min}\{\xi|M_\xi=\xi\}\)

 The standard form for 0 is 0
 If \(\alpha\) is a weakly Mahlo cardinal, then the standard form for \(\alpha\) is \(\alpha= M_\beta\) where \(\beta<\alpha\) and \(\beta\) is expressed in standard form
 If \(\alpha\) is an uncountable regular cardinal of the form \(\chi_\pi(\beta)\), then the standard form for \(\alpha\) is \(\alpha= \chi_\pi(\beta)\) where \(\pi\) is a weakly Mahlo cardinal and \(\pi,\beta\) are expressed in standard form
 If \(\alpha\) is not additively principal and \(\alpha>0\), then the standard form for \(\alpha\) is \(\alpha=\alpha_1+\alpha_2+\cdots+\alpha_n\), where the \(\alpha_i\) are principal ordinals with \(\alpha_1\geq\alpha_2\geq\cdots\geq\alpha_n\), and the \(\alpha_i\) are expressed in standard form
 If \(\alpha\) is an additively principal ordinal but not of the form \(M_\beta\) or \(\chi_\rho(\gamma)\), then \(\alpha\) is expressible in the form \(\psi_\pi(\delta)\). Then the standard form for \(\alpha\) is \(\alpha=\psi_\pi(\delta)\) where \(\pi\) is an uncountable regular cardinal and \(\pi, \delta\) are expressed in standard form

Fundamental sequences for the functions collapsing weakly Mahlo cardinals

The fundamental sequence for an ordinal number \(\alpha\) with cofinality \(\text{cof}(\alpha)=\beta\) is a strictly increasing sequence \((\alpha[\eta])_{\eta<\beta}\) with length \(\beta\) and with limit \(\alpha\), where \(\alpha[\eta]\) is the \(\eta\)-th element of this sequence.

Let \(L=\{\alpha|\text{cof}(\alpha)\geq\omega\}\) denotes the set of all limit ordinals.

For non-zero ordinals \(\alpha<\text{min}\{\xi|M_\xi=\xi\}\) written in the standard form fundamental sequences are defined as follows:

If \(\alpha=\alpha_1+\alpha_2+\cdots+\alpha_n\) with \(n\geq 2\) then \(\text{cof}(\alpha)=\text{cof}(\alpha_n)\) and \(\alpha[\eta]=\alpha_1+\alpha_2+\cdots+(\alpha_n[\eta])\)
 If \(\alpha=\psi_\pi(0)\) then \(\text{cof}(\alpha)=\alpha=1\) and \(\alpha[0]=0\)
 If \(\alpha=\psi_{\chi_\pi(\beta)}(\gamma+1)\) then \(\text{cof}(\alpha)=\omega\) and \(\alpha[\eta]=\left\{\begin{array}{lcr} \chi_\pi(\gamma)\cdot \eta \text{ if }0\le\gamma<\beta\\ \psi_{\chi_\pi(\beta)}(\gamma)\cdot \eta \text{ if }\gamma\geq\beta\\ \end{array}\right.\)
 If \(\alpha=\psi_{M_\beta}(\gamma+1)\) then \(\text{cof}(\alpha)=\omega\) and \(\alpha[\eta]=\chi_{M_\beta}(\gamma)\cdot \eta\)
If \(\alpha=\pi\) then \(\text{cof}(\alpha)=\pi\) and \(\alpha[\eta]=\eta\)
If \(\alpha=M_\beta\) and \(\beta\in L\) then \(\text{cof}(\alpha)=\text{cof}(\beta)\) and \(\alpha[\eta]=M_{\beta[\eta]}\)
 If \(\alpha=\psi_\pi(\beta)\) and \(\omega\le\text{cof}(\beta)<\pi\) then \(\text{cof}(\alpha)=\text{cof}(\beta)\) and \(\alpha[\eta]=\psi_\pi(\beta[\eta])\)
 If \(\alpha=\psi_\pi(\beta)\) where \(\text{cof}(\beta)=\rho\geq\pi\) then \(\text{cof}(\alpha)=\omega\) and \(\alpha[\eta]=\psi_\pi(\beta[\gamma[\eta]])\) with \(\gamma[0]=1\) and \(\gamma[\eta+1]=\left\{\begin{array}{lcr} \psi_{\rho}(\beta[\gamma[\eta]])\text{ if }\rho=\chi_{M_{\delta+1}}(\epsilon)\\ \chi_{\rho}(\beta[\gamma[\eta]])\text{ if }\rho= M_{\delta+1}\\ \end{array}\right.\)

Limit of this notation is \(\nu\). If \(\alpha=\nu\) then \(\text{cof}(\alpha)=\omega\) and \(\alpha[0]=1\) and \(\alpha[\eta+1]=M_{\alpha[\eta]}\)

Another system of fundamental sequences 

For the function, collapsing weakly Mahlo cardinals to countable ordinals, the fundamental sequences also can be defined as follows:
\(C_0= \{0,1\}\)
\(C_{n+1}= \{\alpha+\beta,M_\gamma,\chi_\delta(\epsilon),\psi_\zeta(\eta)|\alpha,\beta,\gamma,\delta,\epsilon,\zeta,\eta\in C_n\wedge\delta\in W\wedge\zeta\in R\}\)
\(L(\alpha)=\text{min}\{n<\omega|\alpha\in C_n\}\)
\(\alpha[n]=\text{max}\{\beta<\alpha|L(\beta)\le L(\alpha)+n\}\)
where \(R\) denotes set of all uncountable regular cardinals and \(W\) denotes set of all weakly Mahlos cardinals.

\end{document}
