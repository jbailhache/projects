\documentclass[8pt]{article}
\title{Les ordinaux transfinis}
\usepackage[left=0.5cm,right=0.5cm,top=0.5cm,bottom=0.5cm]{geometry}
\usepackage[utf8]{inputenc}
\begin{document}

\setlength{\parindent}{0pt}

\vspace{-0.4cm}

\begin{center}
\textbf{LES ORDINAUX TRANSFINIS}

par Jacques Bailhache, Janvier 2018
\end{center}

\vspace{-0.2cm}

Un ordinal est soit 0, soit le successeur d'un ordinal, soit la limite ou la borne supérieure de f(0), f(1), f(2), ...
\vspace{-0.7cm}

\section{Ma notation}
\vspace{-0.4cm}
On part de 0, si on ne voit aucune régularité on prend le successeur, si on voit une régularité, si on a une notation pour cette régularité on l'utilise sinon on l'invente, puis on saute à la limite.
\vspace{-0.6cm}

\section{Notation algébrique}
\vspace{-0.4cm}
On définit les opérations arithmétiques sur les ordinaux :
\vspace{-0.4cm}
\smallskip
\begin{itemize}
     \setlength{\itemsep}{1pt}
     \setlength{\parskip}{0pt}
     \setlength{\parsep}{0pt}
\item addition : \( \alpha+0=\alpha ; \alpha+suc(\beta)=suc(\alpha+\beta); \alpha+lim(f)=lim(n \mapsto \alpha+f(n)) \)
\vspace{-0.1cm}
\item multiplication : \( \alpha \times 0 = \alpha ; \alpha \times suc(\beta) = (\alpha \times \beta) + \alpha ; \alpha \times lim(f) = lim (n \mapsto \alpha \times f(n)) \)
\vspace{-0.1cm}
\item exponentiation : \( \alpha^0 = 1 ; \alpha^{suc(\beta)} = \alpha^\beta \times \alpha ; \alpha^{lim(f)} = lim (n \mapsto \alpha^{f(n)}) \)
\end{itemize}
\vspace{-0.8cm}

\section{Fonctions de Veblen}
\vspace{-0.4cm}
Ces fonctions procèdent par énumération de points fixes : \(\varphi(\ldots,\beta,0,\ldots,0,\gamma) \) représente le \((1+\gamma)^{eme}\) point fixe commun aux fonctions \( \xi \mapsto \varphi(\ldots,\delta,\xi,0,\ldots,0)\) pour tous les \(\delta < \beta\).
\vspace{-0.6cm}

\section{Notation de Simmons}
\vspace{-0.4cm}
\( Fix f z = f^w(z+1)\) = plus petit point fixe de f strictement supérieur à z.

\( Next = Fix (\alpha \mapsto \omega^\alpha) \)

\( [0] h = Fix (a \mapsto h^a 0) \)

\( [1] H h = Fix (a \mapsto H^a h 0) \)

\( [2] H h g = Fix (a \mapsto H^a h g 0) \), etc...

Correspondance avec \(\phi\) de Veblen : \( \phi(1+\alpha,\beta) = ([0]^\alpha Next)^{1+\beta} 0 ; 
 \phi(\alpha,\beta,\gamma) = ([0]^\beta (([1] [0])^\alpha Next))^{1+\gamma} 0 \)

\vspace{-0.6cm}

\section{Fonctions effondrantes ordinales (Ordinal collapsing functions)}
\vspace{-0.4cm}
Ces fonctions utilisent des ordinaux non dénombrables pour définir des ordinaux dénombrables. 

On définit des ensembles d'ordinaux qui peuvent être construits à partir de certains ordinaux et de certaines opérations, puis on définit le plus petit ordinal qui n'appartient pas à cet ensemble, ou le plus petit ordinal qui est plus grand que tous les ordinaux dénombrables de cet ensemble.

Ces fonctions sont des extensions de fonctions sur des ordinaux dénombrables, dont on peut atteindre le point fixe en les appliquant à un ordinal non dénombrable, puis le dépasser en les appliquant à des ordinaux non dénombrables plus grands.

Exemples :
\vspace{-0.4cm}
\smallskip
\begin{itemize}
     \setlength{\itemsep}{1pt}
     \setlength{\parskip}{0pt}
     \setlength{\parsep}{0pt}
\item \(\psi\) de Madore : \(\psi(\alpha) = \varepsilon_\alpha \) si \(\alpha < \zeta_0 ; \psi(\Omega) = \zeta_0 \) qui est le plus petit point fixe de \( \alpha \mapsto \varepsilon_\alpha \).
\vspace{-0.1cm}
\item \(\theta\) de Feferman : \(\theta(\alpha,\beta) = \varphi(\alpha,\beta) \) si \( \alpha < \Gamma_0 \) et \( \beta < \Gamma_0 ; \theta(\Omega,0) = \Gamma_0 \) qui est le plus petit point fixe de \( \alpha \mapsto \varphi(\alpha,0) \).
\vspace{-0.1cm}
\item C de Taranovsky : \( C(\alpha,\beta) = \beta+\omega^\alpha \) si \( \alpha \) est dénombrable; \( C(\Omega_1,0) = \varepsilon_0 \) qui est le plus petit point fixe de \( \alpha \mapsto \omega^\alpha \).
\end{itemize}

\vspace{0.1cm}

\begin{tabular}{|c|c|c|c|c|c|c|c|c|}
\hline
Nom		& Symbole		& Ma notation		& Algébrique			& Veblen			& Simmons			& Madore				& Taranovsky 			\\
\hline
Zero		& 0			& 0			& 0				& 				& 				& 					& 0				\\ \hline
Un		& 1			& suc 0			& 1				& \(\varphi(0,0)\)		& 				& 					& C(0,0)			\\ \hline
Deux		& 2			& suc (suc 0)		& 2				& 				& 				& 					& C(0,C(0,0))			\\ \hline
omega		& \(\omega\)		& H suc 0		& \(\omega\)			& \(\varphi(0,1)\)		& \(\omega\)			& 					& C(1,0)			\\ \hline
		& 			& suc (H suc 0)		& \(\omega+1\)			& 				& 				& 					& C(0,C(1,0))			\\ \hline
		&			& H suc (H suc 0)	& \(\omega\times2\)		&				& 				& 					& C(1,C(1,0))			\\ \hline
		&			& H (H suc) 0		& \(\omega^2\)			& \(\varphi(0,2)\)		& 				& 					& C(C(0,C(0,0)),0)		\\ \hline
		&			& H H suc 0		& \(\omega^\omega\)		& \(\varphi(0,\omega)\)		& 				& 					& C(C(1,0),0)			\\ \hline
		&			& H H H suc 0		& \(\omega^{\omega^\omega}\)	& \(\varphi(0,\omega^\omega)\)	&				&					& C(C(C(1,0),0),0)		\\ \hline
Epsilon zero	& \(\varepsilon_0\)	& \(R_1 H suc\ 0\)	& \(\varepsilon_0\)		& \(\varphi(1,0)\)		& \(Next\ \omega\)		& \(\psi(0)\)				& \(C(\Omega_1,0)\)		\\ \hline
		& 			& \(R_1 (R_1 H) suc\ 0\)& \(\varepsilon_1\)		& \(\varphi(1,1)\)		& 				& \(\psi(1)\)				& \(C(\Omega_1,C(\Omega_1,0)\)	\\ \hline
		& 			& \(H R_1 H suc\ 0\)	& \(\varepsilon_\omega\)	& \(\varphi(1,\omega)\) 	&				& \(\psi(\omega)\)			& \(C(C(0,\Omega_1),0)\)	\\ \hline
		& 			& \(R_1 H R_1 H suc\ 0\)&\(\varepsilon_{\varepsilon_0}\)& \(\varphi(1,\varphi(1,0))\)	&				& \(\psi(\psi(0))\)			& \(C(C(C(\Omega_1,0),\Omega_1),0)\)\\ \hline
Zeta zero	& \(\zeta_0\)		& \(R_2 R_1 H suc\ 0\)	& \(\zeta_0\)			& \(\varphi(2,0)\)		& \([0] Next\ \omega\)		& \(\psi(\Omega)\)			& \(C(C(\Omega_1,\Omega_1),0)\)	\\ \hline
Eta zero	& \(\eta_0\)		& \(R_3 R_2 R_1 H suc\ 0\)& \(\eta_0\)			& \(\varphi(3,0)\)		&				&					& \(C(C(\Omega,C(\Omega,\Omega)),0)\) \\ \
		&			& \(= R_{3 \ldots 1} H suc\ 0\) ?&			&				&				&					&				\\ \hline
		&			& \(R_{\omega \ldots 1} H suc\ 0\) ?&			& \(\varphi(\omega,0)\)		&				&					& \(C(C(C(0,\Omega_1),\Omega_1),0)\) \\ \hline
Feferman	& \(\Gamma_0\)		& \(H (x \mapsto R_{x \ldots 1} H suc\ 0) 0\) ?
								& \(\Gamma_0\)			& \(\varphi(1,0,0)\)		& \([1] [0] Next\ \omega\)	& \(\psi(\Omega^\Omega)\)		& \(C(C(C(\Omega_1,\Omega_1),\) \\ 
-Schütte	&			&			&				& \(=\varphi(2 \mapsto 1)\)	&				& 					& \(\Omega_1),0)\)		\\ \hline
Ackermann	&			&			&				& \(\varphi(1,0,0,0)\)		&				& \(\psi(\Omega^{\Omega^2})\)		&				\\ 
		&			&			&				& \(=\varphi(3 \mapsto 1)\)	&				&					&				\\ \hline
Petit ordinal	&			&			&				& \(\varphi(\omega \mapsto 1)\)	&				& \(\psi(\Omega^{\Omega^\omega})\)	& \(C(\Omega_1^\omega,0)\)	\\
de Veblen	&			&			&				&				&				&					& \(=C(C(C(C(0,\Omega_1), \)	\\ 
		&			&			&				&				&				&					& \(\Omega_1),\Omega_1),0)\)	\\ \hline
Grand ordinal	&			&			&				& + petit ord.	 	 	& \([2] [1] [0] Next\ \omega\)	& \(\psi(\Omega^{\Omega^\Omega})\)	& \(C(\Omega_1^{\Omega_1},0)\)	\\
de Veblen	&			&			&				& non rep.			&				&					& \(=C(C(C(C(\Omega_1,\Omega_1),\) \\ 
		&			&			&				&				&				&					& \( \Omega_1),\Omega_1),0) \)	\\ \hline
Ordinal de	&			&			&				&				& + petit ord.			& \(\psi(\varepsilon_{\Omega+1})\)	& \(C(C(\Omega_2,\Omega_1),0)\)	\\
Bachmann-	&			&			&				&				& non rep.			&					&				\\ 
Howard		&			&			&				&				&				&					&				\\ \hline
  
\end{tabular}


\end{document}

