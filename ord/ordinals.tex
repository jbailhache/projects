\documentclass[10pt]{article}
\title{Transfinite ordinals}
\usepackage[left=0.5cm,right=0.5cm,top=0.5cm,bottom=0.5cm]{geometry}
\usepackage[utf8]{inputenc}
\usepackage{amsfonts}
\usepackage{amsmath}
\begin{document}

\setlength{\parindent}{0pt}

\vspace{-0.4cm}

\begin{center}
\textbf{TRANSFINITE ORDINALS}

by Jacques Bailhache, January-february 2018
\end{center}

\section{Defining transfinite ordinal numbers}

Natural numbers can be represented by sets. Each number is represented by the set of all numbers smaller than it.
\begin{itemize}
     \setlength{\itemsep}{1pt}
     \setlength{\parskip}{0pt}
     \setlength{\parsep}{0pt}
\item \( 0 = \lbrace\rbrace \) (the empty set)
\item \( 1 = \lbrace 0 \rbrace = \lbrace\lbrace\rbrace\rbrace \)
\item \( 2 = \lbrace 0, 1 \rbrace = \lbrace\lbrace\rbrace,\lbrace\lbrace\rbrace\rbrace\rbrace \)
\item \( 3 = \lbrace 0, 1, 2 \rbrace = \lbrace\lbrace\rbrace,\lbrace\lbrace\rbrace\rbrace,\lbrace\lbrace\rbrace,\lbrace\lbrace\rbrace\rbrace\rbrace\rbrace \)
\item ...
\end{itemize}
The successor of a natural number can be defined by \( suc(n) = n+1 = n \cup \lbrace n \rbrace \).

We have \( n \leq p \) if and only if \( n \subseteq p \).

\( \mathbb{N} \) is the set of all natural numbers : \( \mathbb{N} = \lbrace0,1,2,3,\ldots\rbrace \)
The natural numbers can be generalized to what is called "transfinite ordinal numbers", or simply "ordinal numbers" or "ordinals", by considering that infinite sets represent ordinal numbers. \( \mathbb{N} \) considered as an ordinal number is written \( \omega \).
\( \omega \) is the least ordinal which is greater than all the numbers 0, 1, 2, 3, ... We say that \( \omega \) is a limit ordinal and 0, 1, 2, 3, ... is a fundamental sequence of \( \omega \). This is written : \( \omega = sup \lbrace 0, 1, 2, 3, ... \rbrace \) or \( \omega = lim ( n \mapsto n )  \) because the n-th element (starting with 0) of the sequence is n. An ordinal does not have a unique fundamental sequence, for example 1, 2, 3, 4, ... is also a fundamental sequence of \( \omega \), because the least ordinal that is greater than all ordinals of this sequence is also \( \omega \) (more generally the limit ordinal is the same if any number of the least items of a sequence are removed), and the same stands for the sequence 0, 2, 4, 6, ...

Any ordinal can be defined as the least ordinal strictly greater than all ordinals of a set : the empty set for 0, \(\lbrace \alpha \rbrace\) for the successor of \( \alpha \),  \(\lbrace \alpha_0,\alpha_1,\alpha_2,...\rbrace\) for an ordinal with fundamental sequence \(\alpha_0, \alpha_1, \alpha_2, ...\)

The successor can be generalized to transfinite ordinal numbers : \( suc(\omega) = \omega+1 =\omega \cup \lbrace \omega \rbrace = \lbrace 0, 1, 2, 3, \ldots, \omega \rbrace ; suc(suc(\omega)) = \omega+2 = \lbrace 0, 1, 2, 3, \ldots, \omega, \omega+1 \rbrace \) and so on.

Then we can consider the set \( \lbrace 0, 1, 2, 3, \ldots, \omega, \omega+1, omega+2, omega+3, \ldots \rbrace \) which is a limit ordinal, and \( \omega, \omega+1, \omega+2, \omega+3, \ldots \) is a fundamental sequence of this ordinal. This ordinal is \( \omega+\omega = \omega \cdot 2 \) or \( \omega \times 2 \) or \( \omega 2 \).

Then we can go on building greater and greater ordinals : \( \omega \cdot 3, \ldots, \omega \cdot \omega = \omega^2, \omega^3, \ldots, \omega^\omega, \omega^{\omega^\omega}, \ldots \).

\bigskip

The definitions of arithmetical operations can be generalized to ordinals :

\begin{itemize}
     \setlength{\itemsep}{1pt}
     \setlength{\parskip}{0pt}
     \setlength{\parsep}{0pt}
\item addition : \( \alpha+0=\alpha ; \alpha+suc(\beta)=suc(\alpha+\beta); \alpha+lim(f)=lim(n \mapsto \alpha+f(n)) \)
\vspace{-0.1cm}
\item multiplication : \( \alpha \cdot 0 = 0 ; \alpha \cdot suc(\beta) = (\alpha \cdot \beta) + \alpha ; \alpha \cdot lim(f) = lim (n \mapsto \alpha \cdot f(n)) \)
\vspace{-0.1cm}
\item exponentiation : \( \alpha^0 = 1 ; \alpha^{suc(\beta)} = \alpha^\beta \cdot \alpha ; \alpha^{lim(f)} = lim (n \mapsto \alpha^{f(n)}) \)
\end{itemize}

Remark that the addition and the multiplication are not commutative, for example \( 1+\omega = \omega \neq \omega+1 \), because if we take 0, 1, 2, 3, ... as fundamental sequence of \( \omega \), then a fundamental sequence of \( 1+\omega \) is 1+0, 1+1, 1+2, 1+3, ... = 1, 2, 3, 4, ... and the least ordinal which is greater than all ordinals of this sequence is \( \omega \).

\section{Veblen functions}

The next step is the limit or least upper bound of \( \omega, \omega^\omega, \omega^{\omega^\omega}, \ldots \) which is called \( \varepsilon_0 \). Note that we have \( \omega^{\varepsilon_0} = \varepsilon_0 \). We say that \( \varepsilon_0 \) is a fixed point (the least one) of the function \( \alpha \mapsto \omega^\alpha \).

Then we can go on with \( \varepsilon_0+1, \varepsilon_0+2, \ldots, \varepsilon_0+\varepsilon_0 = \varepsilon_0 \cdot 2, \ldots, \varepsilon_0 \cdot \varepsilon_0 = \varepsilon_0^2, \varepsilon_0^{\varepsilon_0}, ...\)

The limit of \( \varepsilon_0, \varepsilon_0^{\varepsilon_0}, \varepsilon_0^{\varepsilon_0^{\varepsilon_0}}, \ldots \) is called \( \varepsilon_1 \). It can be shown that it is also the limit of \( \varepsilon_0+1, \omega^{\varepsilon_0+1}, \omega^{\omega^{\varepsilon_0+1}}, \ldots \) (see proof below).

These two fundamental sequences are examples of two ways of ascending ordinals :
\begin{itemize}
     \setlength{\itemsep}{1pt}
     \setlength{\parskip}{0pt}
     \setlength{\parsep}{0pt}
\item Build greater ordinals by combining known ordinals with arithmetic operations.
\item When we have found a function that, when applied to a given ordinal, gives a greater one (for example \( \alpha \mapsto \omega^\alpha \)), enumerate the fixed points of this function. A fixed point of a function f is a value (for example an ordinal) \( \alpha \) with \( f(\alpha) = \alpha \). Under some conditions (see below), the least fixed point of f is the limit of 0, f(0), f(f(0)), f(f(f(0))), ... If it is called \( \alpha \), the next fixed point is the limit of \( \alpha+1, f(\alpha+1), f(f(\alpha+1)), f(f(f(\alpha+1))), \ldots \).
More generally, the least fixed point of f that is greater or equal to \( \zeta \) is the limit of \( \zeta, f(\zeta), f(f(\zeta)), \ldots\).
The Veblen functions use this method.

\end{itemize}

The required conditions are described for example in http://www.cs.man.ac.uk/~hsimmons/ORDINAL-NOTATIONS/Fruitful.pdf page 8 lemma 3.9 : 

For each fruitful function f and each ordinal \( \zeta, f^\omega(\zeta+1) \) is the least ordinal \( \nu \) such that \( \zeta < \nu = f(\nu) \), or the least fixed point of f that is strictly greater than \( \zeta \) (or greater than or equal to \( \zeta+1 \)). 

\( f^\omega(\zeta+1) \) is the limit of \( \zeta+1, f(\zeta+1), f(f(\zeta+1)), \ldots \). 

A fruitful function is a function that is inflationary, monotone, big, and continuous.

A function f is inflationary if \( \alpha \leq f(\alpha) \), monotone if \( \alpha \leq \beta \Rightarrow f(\alpha) \leq f(\beta) \), big if \( \omega^\alpha \leq f(\alpha) \) except possibly for \( \alpha = 0 \), continuous if f(VA) = Vf[A] where VA is the pointwise supremum of the set A.

\bigskip

We will now prove by induction the equivalence of the two fundamental sequences above.

We will use the notation \( \alpha^{\vdots^{\alpha^\beta}} \) for an an "exponential tower" with \( \alpha \) repeated n times. 


Note that the ordinals respectively limits of the fondamental sequence whose n-th term is \(\varepsilon_0^{\varepsilon_0^{\vdots^{{\varepsilon_0}^\omega}}} \) and the one whose n-th term is \( \varepsilon_0^{\varepsilon_0^{\vdots^{\varepsilon_0^{\varepsilon_0}}}} \) is the same, the least fixed point of the function \( \alpha \mapsto {\varepsilon_0}^\alpha \), which is greater than \( \omega \) and also than \( \varepsilon_0 \).

So we have proved what we want if we prove that, for any n, we have \( \omega^{\omega^{\vdots^{\omega^{\omega^{\varepsilon_0+1}}}}} = \varepsilon_0^{\varepsilon_0^{\vdots^{{\varepsilon_0}^\omega}}} \). 

For n = 0, we have \( \omega^{\omega^{\varepsilon_0+1}} = \omega^{\omega^{\varepsilon_0}\cdot\omega} = \omega^{\varepsilon_0\cdot\omega} = (\omega^{\varepsilon_0})^\omega = {\varepsilon_0}^\omega \). 

Now suppose we have \( \omega^{\omega^{\vdots^{\omega^{\omega^{\varepsilon_0+1}}}}} = \varepsilon_0^{\varepsilon_0^{\vdots^{{\varepsilon_0}^\omega}}} \). 

We must prove the equality for n+1, which can be written \( \omega^{\omega^{\omega^{\vdots^{\omega^{\omega^{\varepsilon_0+1}}}}}} = \varepsilon_0^{\varepsilon_0^{\varepsilon_0^{\vdots^{{\varepsilon_0}^\omega}}}} \). 

We have \( \omega^{\omega^{\omega^{\vdots^{\omega^{\omega^{\varepsilon_0+1}}}}}} = \omega^{\varepsilon_0^{\varepsilon_0^{\vdots^{{\varepsilon_0}^\omega}}}} \) (by our hypothesis) \( = \omega^{\varepsilon_0^{1+\varepsilon_0^{\vdots^{{\varepsilon_0}^\omega}}}} \) (for the same reason than \( 1+\omega = \omega \), see above) \( = \omega^{\varepsilon_0\cdot\varepsilon_0^{\varepsilon_0^{\vdots^{{\varepsilon_0}^\omega}}}} = (\omega^{\varepsilon_0})^{\varepsilon_0^{\varepsilon_0^{\vdots^{{\varepsilon_0}^\omega}}}} = \varepsilon_0^{\varepsilon_0^{\varepsilon_0^{\vdots^{{\varepsilon_0}^\omega}}}} \). QED.


In a similar way, the limit of \( \varepsilon_1, \varepsilon_1^{\varepsilon_1}, \varepsilon_1^{\varepsilon_1^{\varepsilon_1}}, \ldots \) is called \( \varepsilon_2 \) and is also the limit of \( \varepsilon_1+1, \omega^{\varepsilon_1+1}, \omega^{\omega^{\varepsilon_1+1}}, \ldots \).

We can define the same way \( \varepsilon_n \) for any natural number n. Then \( \varepsilon_\omega \) is defined as the limit of \( \varepsilon_0, \varepsilon_1, \varepsilon_2, \varepsilon_3, \ldots \), and \( \varepsilon_{\omega+1} \) as the limit of \( \varepsilon_\omega, \varepsilon_\omega^{\varepsilon_\omega}, \varepsilon_\omega^{\varepsilon_\omega^{\varepsilon_\omega}}, \ldots \) or \( \varepsilon_\omega+1, \omega^{\varepsilon^\omega+1}, \omega^{\omega^{\varepsilon_\omega+1}}, \ldots \).

After comes \( \varepsilon_{\varepsilon_0} \), and the limit of \( \varepsilon_0, \varepsilon_{\varepsilon_0}, \varepsilon_{\varepsilon_{\varepsilon_0}}, \ldots \) which is called \( \zeta_0 \). 
This is the least fixed point of \( \alpha \mapsto \varepsilon_\alpha \). The next one is \( \zeta_1 \) which is the limit of \( \zeta_0+1, \varepsilon_{\zeta_0+1}, \varepsilon_{\varepsilon_{\zeta_0+1}}, \ldots \). 
Then we get \( \zeta_2, \zeta_3, \ldots, \zeta_\omega, \zeta_{\omega+1}, \ldots, \zeta_{\varepsilon_0}, \ldots, \zeta_{\zeta_0}, \ldots, \zeta_{\zeta_{\zeta_0}}, \ldots \).
The limit of \( 0, \zeta_0, \zeta_{\zeta_0}, \zeta_{\zeta_{\zeta_0}}, \ldots \) is called \( \eta_0 \). 

We could go on using successively different greek letters, or define a function \( \varphi \) with two variables by : 

\begin{itemize}
     \setlength{\itemsep}{1pt}
     \setlength{\parskip}{0pt}
     \setlength{\parsep}{0pt}

\item \( \varphi(0,\alpha) = \omega^\alpha \)

\item \( \varphi(1,\alpha) = \varepsilon_\alpha \)

\item \( \varphi(2,\alpha) = \zeta_\alpha \)

\item \( \varphi(3,\alpha) = \eta_\alpha \)

\item \( \varphi(\alpha+1,\beta) \) is the \( (1+\beta) \)-th fixed point of \( \xi \mapsto \varphi(\alpha,\xi) \) .

\end{itemize}

Then we can enumerate the fixed points of the function \( \alpha \mapsto \varphi(\alpha,0) \) and define \( \varphi(1,0,\xi) \) as the \( (1+\xi) \) th fixed point of this function.

More generally, we can define \( \varphi(\alpha_n, \alpha_{n-1}, \ldots, \alpha_1, \alpha_0) \). 

See for example https://en.wikipedia.org/wiki/Veblen\_function :

"Let \(z\) be an empty string or a string consisting of one or more comma-separated zeros \(0,0,...,0\) and \(s\) be an empty string or a string consisting of one or more comma-separated ordinals \(\alpha _{1},\alpha _{2},...,\alpha _{n}\) with \(\alpha _{1}>0\). The binary function \(\varphi (\beta ,\gamma )\) can be written as \(\varphi (s,\beta ,z,\gamma )\) where both \(s\) and \(z\) are empty strings.

The finitary Veblen functions are defined as follows:

\begin{itemize}
     \setlength{\itemsep}{1pt}
     \setlength{\parskip}{0pt}
     \setlength{\parsep}{0pt}

\item \(\varphi (\gamma )=\omega ^{\gamma }\)
\item \(\varphi (z,s,\gamma )=\varphi (s,\gamma )\)
\item if \(\beta >0\), then \(\varphi (s,\beta ,z,\gamma )\) denotes the \((1+\gamma )\)-th common fixed point of the functions \(\xi \mapsto \varphi (s,\delta ,\xi ,z)\) for each \(\delta <\beta\)

\end{itemize}

(...)

The limit of the \(\varphi(1,0,...,0)\) where the number of zeroes ranges over \( \omega \), is sometimes known as the "small" Veblen ordinal.

Every non-zero ordinal \(\alpha\) less than the small Veblen ordinal (SVO) can be uniquely written in normal form for the finitary Veblen function:

\(\alpha =\varphi (s_{1})+\varphi (s_{2})+\cdots +\varphi (s_{k})\)

where

\begin{itemize}
     \setlength{\itemsep}{1pt}
     \setlength{\parskip}{0pt}
     \setlength{\parsep}{0pt}

\item \(k\) is a positive integer
\item \(\varphi (s_{1})\geq \varphi (s_{2})\geq \cdots \geq \varphi (s_{k})\)
\item \(s_{m}\) is a string consisting of one or more comma-separated ordinals \(\alpha _{m,1},\alpha _{m,2},...,\alpha _{m,n_{m}}\) where \(\alpha _{m,1}>0\) and each \(\alpha _{m,i}<\varphi (s_{m})\)

\end{itemize} 

For limit ordinals \(\alpha<SVO\), written in normal form for the finitary Veblen function:

\begin{itemize}
     \setlength{\itemsep}{1pt}
     \setlength{\parskip}{0pt}
     \setlength{\parsep}{0pt}

\item \((\varphi(s_1)+\varphi(s_2)+\cdots+\varphi(s_k))[n]=\varphi(s_1)+\varphi(s_2)+\cdots+\varphi(s_k)[n]\),
\item \(\varphi(\gamma)[n]=\)
\begin{itemize}
     \setlength{\itemsep}{1pt}
     \setlength{\parskip}{0pt}
     \setlength{\parsep}{0pt}
\item n if \( \gamma=1 \)
\item \(\varphi(\gamma-1)\cdot n \) if \( \gamma \) is a successor ordinal
\item \( \varphi(\gamma[n]) \) if \( \gamma \) is a limit ordinal
\end{itemize}
\item \(\varphi(s,\beta,z,\gamma)[0]=0\) and \(\varphi(s,\beta,z,\gamma)[n+1]=\varphi(s,\beta-1,\varphi(s,\beta,z,\gamma)[n],z)\) if \(\gamma=0\) and \(\beta\) is a successor ordinal,
\item \(\varphi(s,\beta,z,\gamma)[0]=\varphi(s,\beta,z,\gamma-1)+1\) and \(\varphi(s,\beta,z,\gamma)[n+1]=\varphi(s,\beta-1,\varphi(s,\beta,z,\gamma)[n],z)\) if \(\gamma\) and \(\beta\) are successor ordinals,
\item \(\varphi(s,\beta,z,\gamma)[n]=\varphi(s,\beta,z,\gamma[n])\) if \(\gamma\) is a limit ordinal,
\item \(\varphi(s,\beta,z,\gamma)[n]=\varphi(s,\beta[n],z,\gamma)\) if \(\gamma=0\) and \(\beta\) is a limit ordinal,
\item \(\varphi(s,\beta,z,\gamma)[n]=\varphi(s,\beta[n],\varphi(s,\beta,z,\gamma-1)+1,z)\) if \(\gamma\) is a successor ordinal and \(\beta\) is a limit ordinal. "

\end{itemize}

The Veblen function can be generalized to transfinitely many variables. Instead of writing the list of all the variable of the Veblen function, we can write only the non zero variables with position as indice, for example \( \varphi(\alpha,0,\beta,\gamma) = \varphi(\alpha_3,\beta_1,\gamma_0) \). We can then generalize the Veblen function by allowing any ordinal as indices, writing for example \( SVO = \varphi(1_\omega) \). 

Schütte brackets or Klammersymbols are another way to write Veblen fuctions with transfinitely many variables. A Schütte bracket consists in a matrix with two lines, with the positions of the variables in the second line in increasing order, and the corresponding values in the first line. This matrix is preceded by the function  \( \xi  \mapsto \varphi(\xi) \). If we take \( \xi \mapsto \omega^\xi \), we get the equivalent of the Veblen function. With this notation, the previous example is written : 

\[
( \xi \mapsto \omega^\xi ) 
  \begin{pmatrix}
    \gamma & \beta & \alpha \\
    0 & 1 & 3
  \end{pmatrix}
\]


\section{Summary}

Any ordinal can be defined as the least ordinal strictly greater than all ordinals of a set : the empty set for 0, \(\lbrace \alpha \rbrace\) for the successor of \( \alpha \),  \(\lbrace \alpha_0,\alpha_1,\alpha_2,...\rbrace\) for an ordinal with fundamental sequence \(\alpha_0, \alpha_1, \alpha_2, ...\)

\vspace{-0.7cm}

\section{Algebraic notation}
\vspace{-0.4cm}
We define the following operations on ordinals :
\vspace{-0.4cm}
\smallskip
\begin{itemize}
     \setlength{\itemsep}{1pt}
     \setlength{\parskip}{0pt}
     \setlength{\parsep}{0pt}
\item addition : \( \alpha+0=\alpha ; \alpha+suc(\beta)=suc(\alpha+\beta); \alpha+lim(f)=lim(n \mapsto \alpha+f(n)) \)
\vspace{-0.1cm}
\item multiplication : \( \alpha \times 0 = 0 ; \alpha \times suc(\beta) = (\alpha \times \beta) + \alpha ; \alpha \times lim(f) = lim (n \mapsto \alpha \times f(n)) \)
\vspace{-0.1cm}
\item exponentiation : \( \alpha^0 = 1 ; \alpha^{suc(\beta)} = \alpha^\beta \times \alpha ; \alpha^{lim(f)} = lim (n \mapsto \alpha^{f(n)}) \)
\end{itemize}
\vspace{-0.8cm}

\section{Veblen functions}
\vspace{-0.4cm}
These functions use fixed points enumaration : \(\varphi(\ldots,\beta,0,\ldots,0,\gamma) \) represents the \((1+\gamma)^{th}\) common fixed point of the functions \( \xi \mapsto \varphi(\ldots,\delta,\xi,0,\ldots,0)\) for all \(\delta < \beta\).
\vspace{-0.6cm}

\section{Simmons notation}
\vspace{-0.4cm}
\( Fix f z = f^w(z+1)\) = least fixed point of f strictly greater than z.

\( Next = Fix (\alpha \mapsto \omega^\alpha) \) 

\( [0] h = Fix (\alpha \mapsto h^\alpha \omega) \) ;
\( [1] h g = Fix (\alpha \mapsto h^\alpha g \omega) \) ;
\( [2] h g f = Fix (\alpha \mapsto h^\alpha g f \omega) \) ; etc...

Correspondence with Veblen's \(\phi\) : \( \phi(1+\alpha,\beta) = ([0]^\alpha Next)^{1+\beta} \omega ; 
 \phi(\alpha,\beta,\gamma) = ([0]^\beta (([1] [0])^\alpha Next))^{1+\gamma} \omega \)


\vspace{-0.6cm}

\section{RHS0 notation}
\vspace{-0.4cm}
We start from 0, if we don(t see any regularity we take the successor, if we see a regularity, if we have a notation for this regularity, we use it, else we invent it, then we jump to the limit.

\( H f x = lim\ x, f x, f (f x), \ldots ; R_1 f g x = lim\ g x, f g x, f f g x, \ldots ; R_2 f g h x = lim\ h x, f g h x, f g f g h x, \ldots \)

Correspondence with Simmons notation : 
\( \ldots, [3] \rightarrow R5, [2] \rightarrow R4, [1] \rightarrow R3, [0] \rightarrow R2, Next \rightarrow R1, \omega \rightarrow H suc\ 0 \)

\vspace{-0.6cm}

\section{Ordinal collapsing functions}
\vspace{-0.4cm}
These functions use uncountable ordinals to define countable ordinals.

We define sets of ordinals that can be built from given ordinals and operations, then we take the least ordinal which is not in this set, or the least ordinal which is greater than all contable ordinals of this set.

These functions are extensions of functions on countable ordinals, whose fixed points can be reached by applying them to an uncountable ordinal.

Examples :
\vspace{-0.4cm}
\smallskip
\begin{itemize}
     \setlength{\itemsep}{1pt}
     \setlength{\parskip}{0pt}
     \setlength{\parsep}{0pt}
\item Madore's \(\psi\) : \(\psi(\alpha) = \varepsilon_\alpha \) if \(\alpha < \zeta_0 ; \psi(\Omega) = \zeta_0 \) which is the least fixed point of \( \alpha \mapsto \varepsilon_\alpha \).
\vspace{-0.1cm}
\item Feferman's \(\theta\) : \(\theta(\alpha,\beta) = \varphi(\alpha,\beta) \) if \( \alpha < \Gamma_0 \) and \( \beta < \Gamma_0 ; \theta(\Omega,0) = \Gamma_0 \) which is the least fixed point of \( \alpha \mapsto \varphi(\alpha,0) \).
\vspace{-0.1cm}
\item Taranovsky's C : \( C(\alpha,\beta) = \beta+\omega^\alpha \) if \( \alpha \) is countable; \( C(\Omega_1,0) = \varepsilon_0 \) which is the least fixed point of \( \alpha \mapsto \omega^\alpha \).
\end{itemize}

\vspace{0.1cm}

\begin{tabular}{|c|c|c|c|c|c|c|c|c|}
\hline
Nom		& Symbole		& Algebraic			& Veblen			& Simmons			& RHS0 		& Madore				& Taranovsky 			\\
\hline
Zero		& 0			& 0				& 				& 				& 0			& 					& 0				\\ \hline
One		& 1			& 1				& \(\varphi(0,0)\)		& 				& suc 0			& 					& C(0,0)			\\ \hline
Two		& 2			& 2				& 				& 				& suc (suc 0)		& 					& C(0,C(0,0))			\\ \hline
Omega		& \(\omega\)		& \(\omega\)			& \(\varphi(0,1)\)		& \(\omega\)			& H suc 0		& 					& C(1,0)			\\ \hline
		& 			& \(\omega+1\)			& 				& 				& suc (H suc 0)		& 					& C(0,C(1,0))			\\ \hline
		&			& \(\omega\times2\)		&				& 				& H suc (H suc 0)	& 					& C(1,C(1,0))			\\ \hline
		&			& \(\omega^2\)			& \(\varphi(0,2)\)		& 				& H (H suc) 0		& 					& C(C(0,C(0,0)),0)		\\ \hline
		&			& \(\omega^\omega\)		& \(\varphi(0,\omega)\)		& 				& H H suc 0		& 					& C(C(1,0),0)			\\ \hline
		&			& \(\omega^{\omega^\omega}\)	& \(\varphi(0,\omega^\omega)\)	&				& H H H suc 0		&					& C(C(C(1,0),0),0)		\\ \hline
Epsilon zero	& \(\varepsilon_0\)	& \(\varepsilon_0\)		& \(\varphi(1,0)\)		& \(Next\ \omega\)		& \(R_1 H suc\ 0\)	& \(\psi(0)\)				& \(C(\Omega_1,0)\)		\\ \hline
		& 			& \(\varepsilon_1\)		& \(\varphi(1,1)\)		& \(Next^2 \omega\)	& \(R_1 (R_1 H) suc\ 0\)& \(\psi(1)\)				& \(C(\Omega_1,C(\Omega_1,0)\)	\\ \hline
		& 			& \(\varepsilon_\omega\)	& \(\varphi(1,\omega)\) 	& \(Next^\omega \omega\) & \(H R_1 H suc\ 0\)	& \(\psi(\omega)\)			& \(C(C(0,\Omega_1),0)\)	\\ \hline
		& 			&\(\varepsilon_{\varepsilon_0}\)& \(\varphi(1,\varphi(1,0))\)	& \(Next^{Next \omega} \omega \) & \(R_1 H R_1 H suc\ 0\)& \(\psi(\psi(0))\)			& \(C(C(C(\Omega_1,0),\Omega_1),0)\)\\ \hline
Zeta zero	& \(\zeta_0\)		& \(\zeta_0\)			& \(\varphi(2,0)\)		& \([0] Next\ \omega\)		& \(R_2 R_1 H suc\ 0\)	& \(\psi(\Omega)\)			& \(C(C(\Omega_1,\Omega_1),0)\)	\\ \hline
Eta zero	& \(\eta_0\)		& \(\eta_0\)			& \(\varphi(3,0)\)		& \([0]^2 Next\ \omega\) 	& \(R_2 (R_2 R_1) H suc\ 0\)&					& \(C(C(\Omega,C(\Omega,\Omega)),0)\) \\ \hline
		&			&			& \(\varphi(\omega,0)\)		& \([0]^\omega Next\ \omega\) & \(H R_2 R_1 H suc\ 0\)&					& \(C(C(C(0,\Omega_1),\Omega_1),0)\) \\ \hline
Feferman	& \(\Gamma_0\)		
								& \(\Gamma_0\)			& \(\varphi(1,0,0)\)		& \([1] [0] Next\ \omega\)	& \(R_3 R_2 R_1 H suc\ 0\) & \(\psi(\Omega^\Omega)\)		& \(C(C(C(\Omega_1,\Omega_1),\) \\ 
-Schütte	&			&				& \(=\varphi(2 \mapsto 1)\)	&				& \(= R_{3 \ldots 1} H suc\ 0\) & 					& \(\Omega_1),0)\)		\\ \hline
Ackermann	&			&				& \(\varphi(1,0,0,0)\)		& \([1]^2 [0] Next\ \omega\) & \(R_3 (R_3 R_2) R_1 H suc\ 0\) & \(\psi(\Omega^{\Omega^2})\)		&				\\ 
		&			&				& \(=\varphi(3 \mapsto 1)\)	&				&			&					&				\\ \hline
Small Veblen	&			&				& \(\varphi(\omega \mapsto 1)\)	& \([1]^\omega [0] Next\ \omega\) & \(H R_3 R_2 R_1 H suc\ 0\) & \(\psi(\Omega^{\Omega^\omega})\)	& \(C(\Omega_1^\omega,0)\)	\\
ordinal		&			&				&				&				&			&					& \(=C(C(C(C(0,\Omega_1), \)	\\ 
		&			&				&				&				&			&					& \(\Omega_1),\Omega_1),0)\)	\\ \hline
Large Veblen	&			&				& least ord.	 	 	& \([2] [1] [0] Next\ \omega\)	& \(R_4 R_3 R_2 R_1 H suc\ 0\) & \(\psi(\Omega^{\Omega^\Omega})\)	& \(C(\Omega_1^{\Omega_1},0)\)	\\
ordinal		&			&				& not rep.			&				& \(= R_{4 \ldots 1} H suc\ 0\) &					& \(=C(C(C(C(\Omega_1,\Omega_1),\) \\ 
		&			&				&				&				&			&					& \( \Omega_1),\Omega_1),0) \)	\\ \hline
Bachmann-	&			&				&				& least ord.			& \(R_{\omega \ldots 1} H suc\ 0\) & \(\psi(\varepsilon_{\Omega+1})\)	& \(C(C(\Omega_2,\Omega_1),0)\)	\\
Howard		&			&				&				& not rep.			&			&					&				\\ 
ordinal		&			&				&				&				&			&					&				\\ \hline
  
\end{tabular}


\end{document}

