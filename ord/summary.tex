\documentclass[8pt]{article}
\title{Transfinite ordinals}
\usepackage[left=0.5cm,right=0.5cm,top=0.5cm,bottom=0.5cm]{geometry}
\usepackage[utf8]{inputenc}
\begin{document}

\setlength{\parindent}{0pt}

\vspace{-0.4cm}

\begin{center}
\textbf{TRANSFINITE ORDINALS}

by Jacques Bailhache, January 2018
\end{center}

\vspace{-0.2cm}

An ordinal is either 0, either the successor of an ordinal, either the limit or least upper bound of f(0), f(1), f(2), ...
\vspace{-0.7cm}

\section{My notation}
\vspace{-0.4cm}
We start from 0, if we don(t see any regularity we take the successor, if we see a regularity, if we have a notation for this regularity, we use it, else we invent it, then we jump to the limit.

\vspace{-0.6cm}

\section{Algebraic notation}
\vspace{-0.4cm}
We define the following operations on ordinals :
\vspace{-0.4cm}
\smallskip
\begin{itemize}
     \setlength{\itemsep}{1pt}
     \setlength{\parskip}{0pt}
     \setlength{\parsep}{0pt}
\item addition : \( \alpha+0=\alpha ; \alpha+suc(\beta)=suc(\alpha+\beta); \alpha+lim(f)=lim(n \mapsto \alpha+f(n)) \)
\vspace{-0.1cm}
\item multiplication : \( \alpha \times 0 = \alpha ; \alpha \times suc(\beta) = (\alpha \times \beta) + \alpha ; \alpha \times lim(f) = lim (n \mapsto \alpha \times f(n)) \)
\vspace{-0.1cm}
\item exponentiation : \( \alpha^0 = 1 ; \alpha^{suc(\beta)} = \alpha^\beta \times \alpha ; \alpha^{lim(f)} = lim (n \mapsto \alpha^{f(n)}) \)
\end{itemize}
\vspace{-0.8cm}

\section{Veblen functions}
\vspace{-0.4cm}
These functions use fixed points enumaration : \(\varphi(\ldots,\beta,0,\ldots,0,\gamma) \) represents the \((1+\gamma)^{th}\) common fixed point of the functions \( \xi \mapsto \varphi(\ldots,\delta,\xi,0,\ldots,0)\) for all \(\delta < \beta\).
\vspace{-0.6cm}

\section{Notation of Simmons}
\vspace{-0.4cm}
\( Fix f z = f^w(z+1)\) = least fixed point of f strictly greater than z.

\( Next = Fix (\alpha \mapsto \omega^\alpha) \)

\( [0] h = Fix (a \mapsto h^a 0) \)

\( [1] H h = Fix (a \mapsto H^a h 0) \)

\( [2] H h g = Fix (a \mapsto H^a h g 0) \), etc...

\vspace{-0.6cm}

\section{Ordinal collapsing functions}
\vspace{-0.4cm}
These functions use uncountable ordinals to define countable ordinals.

We define sets of ordinals that can be built from given ordinals and operations, then we take the least ordinal which is not in this set, or the least ordinal which is greater than all contable ordinals of this set.

These functions are extensions of functions on countable ordinals, whose fixed points can be reached by applying them to an uncountable ordinal.

Examples :
\vspace{-0.4cm}
\smallskip
\begin{itemize}
     \setlength{\itemsep}{1pt}
     \setlength{\parskip}{0pt}
     \setlength{\parsep}{0pt}
\item Madore's \(\psi\) : \(\psi(\alpha) = \varepsilon_\alpha \) if \(\alpha < \zeta_0 ; \psi(\Omega) = \zeta_0 \) which is the least fixed point of \( \alpha \mapsto \varepsilon_\alpha \).
\vspace{-0.1cm}
\item Feferman's \(\theta\) : \(\theta(\alpha,\beta) = \varphi(\alpha,\beta) \) if \( \alpha < \Gamma_0 \) and \( \beta < \Gamma_0 ; \theta(\Omega,0) = \Gamma_0 \) which is the least fixed point of \( \alpha \mapsto \varphi(\alpha,0) \).
\vspace{-0.1cm}
\item Taranovsky's C : \( C(\alpha,\beta) = \beta+\omega^\alpha \) if \( \alpha \) is countable; \( C(\Omega_1,0) = \varepsilon_0 \) which is the least fixed point of \( \alpha \mapsto \omega^\alpha \).
\end{itemize}

\vspace{0.1cm}

\begin{tabular}{|c|c|c|c|c|c|c|c|c|}
\hline
Nom		& Symbole		& Ma notation		& Algébrique			& Veblen			& Simmons			& Madore				& Taranovsky 			\\
\hline
Zero		& 0			& 0			& 0				& 				& 				& 					& 0				\\ \hline
Un		& 1			& suc 0			& 1				& \(\varphi(0,0)\)		& 				& 					& C(0,0)			\\ \hline
Deux		& 2			& suc (suc 0)		& 2				& 				& 				& 					& C(0,C(0,0))			\\ \hline
omega		& \(\omega\)		& H suc 0		& \(\omega\)			& \(\varphi(0,1)\)		& \(\omega\)			& 					& C(1,0)			\\ \hline
		& 			& suc (H suc 0)		& \(\omega+1\)			& 				& 				& 					& C(0,C(1,0))			\\ \hline
		&			& H suc (H suc 0)	& \(\omega\times2\)		&				& 				& 					& C(1,C(1,0))			\\ \hline
		&			& H (H suc) 0		& \(\omega^2\)			& \(\varphi(0,2)\)		& 				& 					& C(C(0,C(0,0)),0)		\\ \hline
		&			& H H suc 0		& \(\omega^\omega\)		& \(\varphi(0,\omega)\)		& 				& 					& C(C(1,0),0)			\\ \hline
		&			& H H H suc 0		& \(\omega^{\omega^\omega}\)	& \(\varphi(0,\omega^\omega)\)	&				&					& C(C(C(1,0),0),0)		\\ \hline
Epsilon zero	& \(\varepsilon_0\)	& \(R_1 H suc\ 0\)	& \(\varepsilon_0\)		& \(\varphi(1,0)\)		& \(Next\ \omega\)		& \(\psi(0)\)				& \(C(\Omega_1,0)\)		\\ \hline
		& 			& \(R_1 (R_1 H) suc\ 0\)& \(\varepsilon_1\)		& \(\varphi(1,1)\)		& 				& \(\psi(1)\)				& \(C(\Omega_1,C(\Omega_1,0)\)	\\ \hline
		& 			& \(H R_1 H suc\ 0\)	& \(\varepsilon_\omega\)	& \(\varphi(1,\omega)\) 	&				& \(\psi(\omega)\)			& \(C(C(0,\Omega_1),0)\)	\\ \hline
		& 			& \(R_1 H R_1 H suc\ 0\)&\(\varepsilon_{\varepsilon_0}\)& \(\varphi(1,\varphi(1,0))\)	&				& \(\psi(\psi(0))\)			& \(C(C(C(\Omega_1,0),\Omega_1),0)\)\\ \hline
Zeta zero	& \(\zeta_0\)		& \(R_2 R_1 H suc\ 0\)	& \(\zeta_0\)			& \(\varphi(2,0)\)		& \([0] Next\ \omega\)		& \(\psi(\Omega)\)			& \(C(C(\Omega_1,\Omega_1),0)\)	\\ \hline
Eta zero	& \(\eta_0\)		& \(R_3 R_2 R_1 H suc\ 0\)& \(\eta_0\)			& \(\varphi(3,0)\)		&				&					& \(C(C(\Omega,C(\Omega,\Omega)),0)\) \\ \
		&			& \(= R_{3 \ldots 1} H suc\ 0\)&			&				&				&					&				\\ \hline
		&			& \(R_{\omega \ldots 1} H suc\ 0\)&			& \(\varphi(\omega,0)\)		&				&					& \(C(C(C(0,\Omega_1),\Omega_1),0)\) \\ \hline
Feferman	& \(\Gamma_0\)		& \(H (x \mapsto R_{x \ldots 1} H suc\ 0) 0\)
								& \(\Gamma_0\)			& \(\varphi(1,0,0)\)		& \([1] [0] Next\ \omega\)	& \(\psi(\Omega^\Omega)\)		& \(C(C(C(\Omega_1,\Omega_1),\) \\ 
-Schütte	&			&			&				& \(=\varphi(2 \mapsto 1)\)	&				& 					& \(\Omega_1),0)\)		\\ \hline
Ackermann	&			&			&				& \(\varphi(1,0,0,0)\)		&				& \(\psi(\Omega^{\Omega^2})\)		&				\\ 
		&			&			&				& \(=\varphi(3 \mapsto 1)\)	&				&					&				\\ \hline
Small Veblen	&			&			&				& \(\varphi(\omega \mapsto 1)\)	&				& \(\psi(\Omega^{\Omega^\omega})\)	& \(C(\Omega_1^\omega,0)\)	\\
ordinal		&			&			&				&				&				&					& \(=C(C(C(C(0,\Omega_1), \)	\\ 
		&			&			&				&				&				&					& \(\Omega_1),\Omega_1),0)\)	\\ \hline
Large Veblen	&			&			&				& least ord.	 	 	& \([2] [1] [0] Next\ \omega\)	& \(\psi(\Omega^{\Omega^\Omega})\)	& \(C(\Omega_1^{\Omega_1},0)\)	\\
ordinal		&			&			&				& not rep.			&				&					& \(=C(C(C(C(\Omega_1,\Omega_1),\) \\ 
		&			&			&				&				&				&					& \( \Omega_1),\Omega_1),0) \)	\\ \hline
Bachmann-	&			&			&				&				& least ord.			& \(\psi(\varepsilon_{\Omega+1})\)	& \(C(C(\Omega_2,\Omega_1),0)\)	\\
Howard		&			&			&				&				& not rep.			&					&				\\ 
ordinal		&			&			&				&				&				&					&				\\ \hline
  
\end{tabular}


\end{document}

